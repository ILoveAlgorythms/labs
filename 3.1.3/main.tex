% !TeX root = main.tex
\documentclass[11pt,a4paper]{article}
\input{../preamble.tex}
\newcommand{\mk}{\mathfrak}
%%% значок в itemize
\renewcommand{\labelitemi}{$\multimap  $}
\title{\texttt{Измерение магнитного поля Земли \\ 3.1.3}}
\author{Кирилл Нелюбин, Б05-207}
\date{\today}

\begin{document}
\maketitle

\section*{Цель работы:}
исследовать свойства постоянных неодимовых магнитов;
измерить с их помощью горизонтальную и вертикальную составляющие
индукции магнитного поля Земли и магнитное наклонение.
\section*{В работе используются:}  
неодимовые магниты; тонкая нить для изготов­
ления крутильного маятника; медная проволока; электронные весы; секун­
домер; измеритель магнитной индукции; штангенциркуль; брусок, линейка
и штатив из немагнитных материалов; набор гирь и разновесов.

% \section*{Экспериментальная установка}
% \subsection*{\texttt{Важные константы:}}

\section*{Ход работы}
\subsection*{Определение магнитного момента магнитных шариков}
\subsubsection{Метод А}
\begin{enumerate}
  % \setcounter{enumi}{-1}
  \item Измерим димаетр шариков и взесим их на весах. Таблица:
  \begin{figure}[H]
    \resizebox{0.7\textwidth}{!}{
      \begin{tabular}{|*{11}{c|}}
        \hline
        d, см $\pm 0.01$& 0.60 & 0.59 & 0.60 & 0.59 & 0.58 & 0.59 & 0.59 & 0.60 & 0.60 & 0.60 \\
        \hline
        \multicolumn{11}{|c|}{$d_\text{ср} = 0.593 \pm 0.001 \text{см}$} \\
        \hline
      \end{tabular}
      }
    \label{tab:diametrs}
    \caption{Диаметры шариков}
  \end{figure}
  Масса 30 шариков: $24.923 \pm 0.003 \text{г} => $ масса шарика $m=0.8308 \pm\text{г}$
  \item Проложим между 2 шариками стопку из бумаги и будем медленно увеличивать её, пока равновесие ещё сохраняется. Максимальное расстояние: $r_{max} = 1.785 \pm 0.003$ мм
  \begin{figure}[H]
    \includegraphics*[width=0.2\textwidth]{2023-10-01-23-50-52.png}
    \caption{Измерение магнитных моментов шариков}
    \label{ust:met_a}
  \end{figure}
  По формуле:
  $$\mathfrak{m}=\sqrt{\frac{mgr^4_{max}}{6}} \text{ ед. СГС}$$
  Получаем $\mk m = $
\end{enumerate}
\subsubsection{Метод Б}
\begin{enumerate}
  % \setcounter{enumi}{-1}
  \item \begin{figure}[H]
    \includegraphics*[width=0.5\textwidth]{2023-10-02-00-43-23.png}
    \label{ust:met_b}
    \caption{Альтернативный ме­тод измерения магнитных
    моментов шариков}
  \end{figure}
  Учитывая, что
  сила в зависимости от расстояния между центрами шаров притяжения убывает как $F\varpropto r^{-4}$, будем учитывать взаимодействие верхнего шара только с 3-4 блийжайшими соседями. Сила сцепления двух одинаковых шаров радиусами $R$ c магнитными моментами $\mathfrak{m}$ равна $$F = \sum_{i=1}^{4} \frac{6\mk m^2}{(2iR)^4} = \underline{\frac{3.24}{8}\cdot\frac{\mathfrak{m}^4}{R^4}}\text{ ед. СГС}$$
\end{enumerate}



\subsection{Измерение горизонтальной составляющей индукции магнитного
поля Земли}\begin{enumerate}
  \item \begin{figure}[H]
    \includegraphics*[width=0.5\textwidth]{2023-10-02-17-27-25.png}
    \caption{Крутильный маятник во внешнем магнитном поле}
  \end{figure}
  \item Период колебаний такого маятника (формула берётся из уравнения колебаний, предполагая, что магнитный момент адитивен):
  $$T(n) = 2\pi \sqrt{\frac{mR^2}{3\mk m B_\Vert }} \cdot n$$
\end{enumerate}



\subsection{Измерение вертикальной составляющей индукции магнитного поля
Земли. Магнитное наклонение.}
\begin{enumerate}
  \item \begin{figure}[H]
    \includegraphics*[width=0.7\textwidth]{2023-10-02-17-36-48.png}
    \caption{Измерение вертикальной составляющей поля и магнитного наклонения}
  \end{figure}
  \item Формула для $B_\bot:$
  $$\mathcal{M}_n = m_\text{гр}gr_\text{гр} = n \mk m B_\bot$$
\end{enumerate}

\section*{Вывод}
Энергия активации примерно свопадает с табличным значением.
Как видно из графиков, число Рейнольдса зависит от температуры.
Время релаксации тоже зависит, и как видно, для стеклянных и стальных шариков оно сильно отличается
\end{document}
