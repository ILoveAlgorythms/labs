% !TeX root = main.tex
\documentclass[11pt,a4paper]{article}
% \documentclass[a4paper,14pt, draft]{article}

%%% отключение нумерации сраниц
\pagestyle{empty}
%%% значок в itemize
% \renewcommand{\labelitemi}{$\cdot$}

%%% Работа с русским языком
\usepackage{cmap}					% поиск в PDF
\usepackage{mathtext} 				% русские буквы в формулах
\usepackage[T1, T2A]{fontenc}			% кодировка %Т1 посоветовал чат гпт
\usepackage[utf8]{inputenc}			% кодировка исходного текста
\usepackage[english,russian]{babel}	% локализация и переносы
\usepackage{indentfirst}            % красная строка в первом абзаце
\frenchspacing                      % равные пробелы между словами и предложениями

%%% Дополнительная работа с математикой
\usepackage{amsmath,amsfonts,amssymb,amsthm,mathtools} % пакеты AMS
\usepackage{icomma}                                    % "Умная" запятая

%%% Свои символы и команды
\usepackage{centernot} % центрированное зачеркивание символа
\usepackage{stmaryrd}  % некоторые спецсимволы
\usepackage{dsfont}
\usepackage{amsthm}


\renewcommand{\epsilon}{\ensuremath{\varepsilon}}
\renewcommand{\phi}{\ensuremath{\varphi}}
\renewcommand{\kappa}{\ensuremath{\varkappa}}
\renewcommand{\le}{\ensuremath{\leqslant}}
\renewcommand{\leq}{\ensuremath{\leqslant}}
\renewcommand{\ge}{\ensuremath{\geqslant}}
\renewcommand{\geq}{\ensuremath{\geqslant}}
\renewcommand{\emptyset}{\ensuremath{\varnothing}}

\DeclareMathOperator{\sgn}{sgn}
\DeclareMathOperator{\ke}{Ker}
\DeclareMathOperator{\im}{Im}
\DeclareMathOperator{\re}{Re}

\newcommand{\N}{\mathbb{N}}
\newcommand{\Z}{\mathbb{Z}}
\newcommand{\Q}{\mathbb{Q}}
\newcommand{\R}{\mathbb{R}}
\newcommand{\Cm}{\mathbb{C}}
\newcommand{\F}{\mathbb{F}}
\newcommand{\id}{\mathrm{id}}
\newcommand{\imp}[2]{
	(#1\,\,$\ra$\,\,#2)\,\,
}
\newcommand{\Root}[2]{
	\left\{\!\sqrt[#1]{#2}\right\}
}
\newcommand{\RR}{\R}
\newcommand{\NN}{\N}
\renewcommand{\subseteq}{\subset}
\newcommand{\sub}{\subset}
\newcommand{\sconstr}{\;\vert\;}
\newcommand{\thus}{\implies}

\newcommand{\defeq}{\vcentcolon= }
\newcommand{\defev}{\stackrel{\Delta}{\Longleftrightarrow}}
\newcommand{\deriv}[3][1]{%
	\ifthenelse{#1>1}{%
		\frac{\dlta^{#1} {#2}}{\dlta {#3}^{#1}}
	}{%
		\frac{\dlta {#2}}{\dlta {#3}}
	}%
}

\renewcommand\labelitemi{$\triangleright$}

\let\bs\backslash
\let\lra\Leftrightarrow
\let\ra\Rightarrow
\let\la\Leftarrow
\let\emb\hookrightarrow

%%% Перенос знаков в формулах (по Львовскому)
\newcommand{\hm}[1]{#1\nobreak\discretionary{}{\hbox{$\mathsurround=0pt #1$}}{}}

%%% Работа с картинками
\usepackage{graphicx}    % Для вставки рисунков
\setlength\fboxsep{3pt}  % Отступ рамки \fbox{} от рисунка
\setlength\fboxrule{1pt} % Толщина линий рамки \fbox{}
\usepackage{wrapfig}     % Обтекание рисунков текстом

% \usepackage[inkscapeformat=png]{svg} %% svg

%%% Работа с таблицами
\usepackage{array,tabularx,tabulary,booktabs} % Дополнительная работа с таблицами
\usepackage{longtable}                        % Длинные таблицы
\usepackage{multirow}                         % Слияние строк в таблице

%%% Теоремы
\theoremstyle{plain}
\newtheorem*{theorem}{Теорема}
\newtheorem*{lemma}{Лемма}
\newtheorem*{proposition}{Утверждение}
\newtheorem*{exercise}{Упражнение}
\newtheorem*{problem}{Задача}

\theoremstyle{definition}
\newtheorem*{definition}{Определение}
\newtheorem*{corollary}{Следствие}
\newtheorem*{note}{Замечание}
\newtheorem*{reminder}{Напоминание}
\newtheorem*{example}{Пример}

\theoremstyle{remark}
\newtheorem*{solution}{Решение}

%%% Оформление страницы
\usepackage{extsizes}     % Возможность сделать 14-й шрифт
\usepackage{geometry}     % Простой способ задавать поля
\usepackage{setspace}     % Интерлиньяж
\usepackage{enumitem}     % Настройка окружений itemize и enumerate
\setlist{leftmargin=10pt} % Отступы в itemize и enumerate

\geometry{top=15mm}    % Поля сверху страницы
\geometry{bottom=5mm} % Поля снизу страницы
\geometry{left=10mm}   % Поля слева страницы
\geometry{right=10mm}  % Поля справа страницы

\setlength\parindent{15pt}        % Устанавливает длину красной строки 15pt
\linespread{1}                  % Коэффициент межстрочного интервала
%\setlength{\parskip}{0.5em}      % Вертикальный интервал между абзацами
\setcounter{secnumdepth}{0}      % Отключение нумерации разделов
%\setcounter{section}{-1}         % Нумерация секций с нуля
\usepackage{multicol}			  % Для текста в нескольких колонках
\usepackage{soulutf8}             % Модификаторы начертания
\mathtoolsset{showonlyrefs=true} % показывать номера формул только у тех, у которых есть ссылки по eqref
%%% Содержаниие
% \usepackage{tocloft}
% \tocloftpagestyle{main}
%\setlength{\cftsecnumwidth}{2.3em}
%\renewcommand{\cftsecdotsep}{1}
%\renewcommand{\cftsecpresnum}{\hfill}
%\renewcommand{\cftsecaftersnum}{\quad}

%%% Нумерация уравнений
\makeatletter
\def\eqref{\@ifstar\@eqref\@@eqref}
\def\@eqref#1{\textup{\tagform@{\ref*{#1}}}}
\def\@@eqref#1{\textup{\tagform@{\ref{#1}}}}
\makeatother                      % \eqref* без гиперссылки
\numberwithin{equation}{section}  % Нумерация вида (номер_секции).(номер_уравнения)
\mathtoolsset{showonlyrefs= true} % Номера только у формул с \eqref{} в тексте.

%%% Гиперссылки
\usepackage{hyperref}
\usepackage[usenames,dvipsnames,svgnames,table,rgb]{xcolor}
\hypersetup{
	unicode=true,            % русские буквы в раздела PDF
	colorlinks=true,       	 % Цветные ссылки вместо ссылок в рамках
	linkcolor=black!15!blue, % Внутренние ссылки
	citecolor=green,         % Ссылки на библиографию
	filecolor=magenta,       % Ссылки на файлы
	urlcolor=NavyBlue,       % Ссылки на URL
}

%%% Графика
\usepackage{tikz}        % Графический пакет tikz
\usepackage{tikz-cd}     % Коммутативные диаграммы
\usepackage{tkz-euclide} % Геометрия
\usepackage{stackengine} % Многострочные тексты в картинках
\usetikzlibrary{angles, babel, quotes}
\newcommand{\mk}{\mathfrak}
%%% значок в itemize
\renewcommand{\labelitemi}{$\multimap  $}
\title{\texttt{Измерение магнитного поля Земли \\ 3.1.3}}
\author{Кирилл Нелюбин, Б05-207}
\date{\today}

\begin{document}
\maketitle

\section*{Цель работы:}
исследовать свойства постоянных неодимовых магнитов;
измерить с их помощью горизонтальную и вертикальную составляющие
индукции магнитного поля Земли и магнитное наклонение.
\section*{В работе используются:}  
неодимовые магниты; тонкая нить для изготов­
ления крутильного маятника; медная проволока; электронные весы; секун­
домер; измеритель магнитной индукции; штангенциркуль; брусок, линейка
и штатив из немагнитных материалов; набор гирь и разновесов.

% \section*{Экспериментальная установка}
% \subsection*{\texttt{Важные константы:}}

\section*{Ход работы}
\subsection*{Определение магнитного момента магнитных шариков}
\subsubsection{Метод А}
\begin{enumerate}
  % \setcounter{enumi}{-1}
  \item Измерим димаетр шариков и взесим их на весах. Таблица:
  \begin{figure}[H]
    \resizebox{0.7\textwidth}{!}{
      \begin{tabular}{|*{11}{c|}}
        \hline
        d, см $\pm 0.01$& 0.60 & 0.59 & 0.60 & 0.59 & 0.58 & 0.59 & 0.59 & 0.60 & 0.60 & 0.60 \\
        \hline
        \multicolumn{11}{|c|}{$d_\text{ср} = 0.593 \pm 0.001 \text{см}$} \\
        \hline
      \end{tabular}
      }
    \label{tab:diametrs}
    \caption{Диаметры шариков}
  \end{figure}
  Масса 30 шариков: $24.923 \pm 0.003 \text{г} => $ масса шарика $m=0.8308 \pm\text{г}$
  \item Проложим между 2 шариками стопку из бумаги и будем медленно увеличивать её, пока равновесие ещё сохраняется. Максимальное расстояние: $r_{max} = 1.785 \pm 0.003$ мм
  \begin{figure}[H]
    \includegraphics*[width=0.2\textwidth]{2023-10-01-23-50-52.png}
    \caption{Измерение магнитных моментов шариков}
    \label{ust:met_a}
  \end{figure}
  По формуле:
  $$\mathfrak{m}=\sqrt{\frac{mgr^4_{max}}{6}} \text{ ед. СГС}$$
  Получаем $\mk m = $
\end{enumerate}
\subsubsection{Метод Б}
\begin{enumerate}
  % \setcounter{enumi}{-1}
  \item \begin{figure}[H]
    \includegraphics*[width=0.5\textwidth]{2023-10-02-00-43-23.png}
    \label{ust:met_b}
    \caption{Альтернативный ме­тод измерения магнитных
    моментов шариков}
  \end{figure}
  Учитывая, что
  сила в зависимости от расстояния между центрами шаров притяжения убывает как $F\varpropto r^{-4}$, будем учитывать взаимодействие верхнего шара только с 3-4 блийжайшими соседями. Сила сцепления двух одинаковых шаров радиусами $R$ c магнитными моментами $\mathfrak{m}$ равна $$F = \sum_{i=1}^{4} \frac{6\mk m^2}{(2iR)^4} = \underline{\frac{3.24}{8}\cdot\frac{\mathfrak{m}^4}{R^4}}\text{ ед. СГС}$$
\end{enumerate}



\subsection{Измерение горизонтальной составляющей индукции магнитного
поля Земли}\begin{enumerate}
  \item \begin{figure}[H]
    \includegraphics*[width=0.5\textwidth]{2023-10-02-17-27-25.png}
    \caption{Крутильный маятник во внешнем магнитном поле}
  \end{figure}
  \item Период колебаний такого маятника (формула берётся из уравнения колебаний, предполагая, что магнитный момент адитивен):
  $$T(n) = 2\pi \sqrt{\frac{mR^2}{3\mk m B_\Vert }} \cdot n$$
\end{enumerate}



\subsection{Измерение вертикальной составляющей индукции магнитного поля
Земли. Магнитное наклонение.}
\begin{enumerate}
  \item \begin{figure}[H]
    \includegraphics*[width=0.7\textwidth]{2023-10-02-17-36-48.png}
    \caption{Измерение вертикальной составляющей поля и магнитного наклонения}
  \end{figure}
  \item Формула для $B_\bot:$
  $$\mathcal{M}_n = m_\text{гр}gr_\text{гр} = n \mk m B_\bot$$
\end{enumerate}

\section*{Вывод}
Энергия активации примерно свопадает с табличным значением.
Как видно из графиков, число Рейнольдса зависит от температуры.
Время релаксации тоже зависит, и как видно, для стеклянных и стальных шариков оно сильно отличается
\end{document}
