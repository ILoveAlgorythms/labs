% !TeX root = main.tex
\documentclass[12pt,a4paper]{article}
% \documentclass[a4paper,14pt, draft]{article}

%%% отключение нумерации сраниц
\pagestyle{empty}
%%% значок в itemize
% \renewcommand{\labelitemi}{$\cdot$}

%%% Работа с русским языком
\usepackage{cmap}					% поиск в PDF
\usepackage{mathtext} 				% русские буквы в формулах
\usepackage[T1, T2A]{fontenc}			% кодировка %Т1 посоветовал чат гпт
\usepackage[utf8]{inputenc}			% кодировка исходного текста
\usepackage[english,russian]{babel}	% локализация и переносы
\usepackage{indentfirst}            % красная строка в первом абзаце
\frenchspacing                      % равные пробелы между словами и предложениями

%%% Дополнительная работа с математикой
\usepackage{amsmath,amsfonts,amssymb,amsthm,mathtools} % пакеты AMS
\usepackage{icomma}                                    % "Умная" запятая

%%% Свои символы и команды
\usepackage{centernot} % центрированное зачеркивание символа
\usepackage{stmaryrd}  % некоторые спецсимволы
\usepackage{dsfont}
\usepackage{amsthm}


\renewcommand{\epsilon}{\ensuremath{\varepsilon}}
\renewcommand{\phi}{\ensuremath{\varphi}}
\renewcommand{\kappa}{\ensuremath{\varkappa}}
\renewcommand{\le}{\ensuremath{\leqslant}}
\renewcommand{\leq}{\ensuremath{\leqslant}}
\renewcommand{\ge}{\ensuremath{\geqslant}}
\renewcommand{\geq}{\ensuremath{\geqslant}}
\renewcommand{\emptyset}{\ensuremath{\varnothing}}

\DeclareMathOperator{\sgn}{sgn}
\DeclareMathOperator{\ke}{Ker}
\DeclareMathOperator{\im}{Im}
\DeclareMathOperator{\re}{Re}

\newcommand{\N}{\mathbb{N}}
\newcommand{\Z}{\mathbb{Z}}
\newcommand{\Q}{\mathbb{Q}}
\newcommand{\R}{\mathbb{R}}
\newcommand{\Cm}{\mathbb{C}}
\newcommand{\F}{\mathbb{F}}
\newcommand{\id}{\mathrm{id}}
\newcommand{\imp}[2]{
	(#1\,\,$\ra$\,\,#2)\,\,
}
\newcommand{\Root}[2]{
	\left\{\!\sqrt[#1]{#2}\right\}
}
\newcommand{\RR}{\R}
\newcommand{\NN}{\N}
\renewcommand{\subseteq}{\subset}
\newcommand{\sub}{\subset}
\newcommand{\sconstr}{\;\vert\;}
\newcommand{\thus}{\implies}

\newcommand{\defeq}{\vcentcolon= }
\newcommand{\defev}{\stackrel{\Delta}{\Longleftrightarrow}}
\newcommand{\deriv}[3][1]{%
	\ifthenelse{#1>1}{%
		\frac{\dlta^{#1} {#2}}{\dlta {#3}^{#1}}
	}{%
		\frac{\dlta {#2}}{\dlta {#3}}
	}%
}

\renewcommand\labelitemi{$\triangleright$}

\let\bs\backslash
\let\lra\Leftrightarrow
\let\ra\Rightarrow
\let\la\Leftarrow
\let\emb\hookrightarrow

%%% Перенос знаков в формулах (по Львовскому)
\newcommand{\hm}[1]{#1\nobreak\discretionary{}{\hbox{$\mathsurround=0pt #1$}}{}}

%%% Работа с картинками
\usepackage{graphicx}    % Для вставки рисунков
\setlength\fboxsep{3pt}  % Отступ рамки \fbox{} от рисунка
\setlength\fboxrule{1pt} % Толщина линий рамки \fbox{}
\usepackage{wrapfig}     % Обтекание рисунков текстом

% \usepackage[inkscapeformat=png]{svg} %% svg

%%% Работа с таблицами
\usepackage{array,tabularx,tabulary,booktabs} % Дополнительная работа с таблицами
\usepackage{longtable}                        % Длинные таблицы
\usepackage{multirow}                         % Слияние строк в таблице

%%% Теоремы
\theoremstyle{plain}
\newtheorem*{theorem}{Теорема}
\newtheorem*{lemma}{Лемма}
\newtheorem*{proposition}{Утверждение}
\newtheorem*{exercise}{Упражнение}
\newtheorem*{problem}{Задача}

\theoremstyle{definition}
\newtheorem*{definition}{Определение}
\newtheorem*{corollary}{Следствие}
\newtheorem*{note}{Замечание}
\newtheorem*{reminder}{Напоминание}
\newtheorem*{example}{Пример}

\theoremstyle{remark}
\newtheorem*{solution}{Решение}

%%% Оформление страницы
\usepackage{extsizes}     % Возможность сделать 14-й шрифт
\usepackage{geometry}     % Простой способ задавать поля
\usepackage{setspace}     % Интерлиньяж
\usepackage{enumitem}     % Настройка окружений itemize и enumerate
\setlist{leftmargin=10pt} % Отступы в itemize и enumerate

\geometry{top=15mm}    % Поля сверху страницы
\geometry{bottom=5mm} % Поля снизу страницы
\geometry{left=10mm}   % Поля слева страницы
\geometry{right=10mm}  % Поля справа страницы

\setlength\parindent{15pt}        % Устанавливает длину красной строки 15pt
\linespread{1}                  % Коэффициент межстрочного интервала
%\setlength{\parskip}{0.5em}      % Вертикальный интервал между абзацами
\setcounter{secnumdepth}{0}      % Отключение нумерации разделов
%\setcounter{section}{-1}         % Нумерация секций с нуля
\usepackage{multicol}			  % Для текста в нескольких колонках
\usepackage{soulutf8}             % Модификаторы начертания
\mathtoolsset{showonlyrefs=true} % показывать номера формул только у тех, у которых есть ссылки по eqref
%%% Содержаниие
% \usepackage{tocloft}
% \tocloftpagestyle{main}
%\setlength{\cftsecnumwidth}{2.3em}
%\renewcommand{\cftsecdotsep}{1}
%\renewcommand{\cftsecpresnum}{\hfill}
%\renewcommand{\cftsecaftersnum}{\quad}

%%% Нумерация уравнений
\makeatletter
\def\eqref{\@ifstar\@eqref\@@eqref}
\def\@eqref#1{\textup{\tagform@{\ref*{#1}}}}
\def\@@eqref#1{\textup{\tagform@{\ref{#1}}}}
\makeatother                      % \eqref* без гиперссылки
\numberwithin{equation}{section}  % Нумерация вида (номер_секции).(номер_уравнения)
\mathtoolsset{showonlyrefs= true} % Номера только у формул с \eqref{} в тексте.

%%% Гиперссылки
\usepackage{hyperref}
\usepackage[usenames,dvipsnames,svgnames,table,rgb]{xcolor}
\hypersetup{
	unicode=true,            % русские буквы в раздела PDF
	colorlinks=true,       	 % Цветные ссылки вместо ссылок в рамках
	linkcolor=black!15!blue, % Внутренние ссылки
	citecolor=green,         % Ссылки на библиографию
	filecolor=magenta,       % Ссылки на файлы
	urlcolor=NavyBlue,       % Ссылки на URL
}

%%% Графика
\usepackage{tikz}        % Графический пакет tikz
\usepackage{tikz-cd}     % Коммутативные диаграммы
\usepackage{tkz-euclide} % Геометрия
\usepackage{stackengine} % Многострочные тексты в картинках
\usetikzlibrary{angles, babel, quotes}
\newcommand{\mk}{\mathfrak}
%%% значок в itemize
\renewcommand{\labelitemi}{$\multimap  $}
\title{\texttt{Измерение магнитного поля Земли \\ 3.1.3}}
\author{Кирилл Нелюбин, Б05-207}
\date{\today}

\begin{document}
\maketitle

\section{Цель работы:}
исследовать свойства постоянных неодимовых магнитов;
измерить с их помощью горизонтальную и вертикальную составляющие
индукции магнитного поля Земли и магнитное наклонение.
\section{В работе используются:}  
неодимовые магниты; тонкая нить для изготов­
ления крутильного маятника; медная проволока; электронные весы; секун­
домер; измеритель магнитной индукции; штангенциркуль; брусок, линейка
и штатив из немагнитных материалов; набор гирь и разновесов.

% \section*{Экспериментальная установка}
% \subsection*{\texttt{Важные константы:}}

\section{Ход работы}
\subsection{Определение магнитного момента магнитных шариков}
\subsubsection{Метод А}
\begin{enumerate}
  % \setcounter{enumi}{-1}
  \item Измерим димаетр шариков и взесим их на весах. Таблица:
  \begin{figure}[H]
    \resizebox{0.7\textwidth}{!}{
      \begin{tabular}{|*{11}{c|}}
        \hline
        d, см $\pm 0.01$& 0.60 & 0.59 & 0.60 & 0.59 & 0.58 & 0.59 & 0.59 & 0.60 & 0.60 & 0.60 \\
        \hline
        \multicolumn{11}{|c|}{$d_\text{ср} = 0.593 \pm 0.001 \text{см}$} \\
        \hline
      \end{tabular}
      }
    \label{tab:diametrs}
    \caption{Диаметры шариков}
  \end{figure}
  Масса 30 шариков: $24.923 \pm 0.003 \text{г} => $ масса шарика $m=0.8308 \pm 0.0001\text{г}$
  \item Проложим между 2 шариками стопку из бумаги и будем медленно увеличивать её, пока равновесие ещё сохраняется. Максимальное расстояние: $r_{max} = 1.785 \pm 0.003$ см
  \begin{figure}[H]
    \includegraphics*[width=0.2\textwidth]{2023-10-01-23-50-52.png}
    \caption{Измерение магнитных моментов шариков}
    \label{ust:met_a}
  \end{figure}
  По формуле \scriptsize(из 2 закона Ньютона)\normalsize:
  $$\mathfrak{m}=\sqrt{\frac{mgr^4_{max}}{6}} \text{ ед. СГС}$$
  Получаем \underline{$\mk m = 37.13 \pm 0.13$ ед. СГС}
\end{enumerate}

\subsubsection{Метод Б}
\begin{enumerate}
  % \setcounter{enumi}{-1}
  \item \begin{figure}[H]
    \includegraphics*[width=0.5\textwidth]{2023-10-02-00-43-23.png}
    \label{ust:met_b}
    \caption{Альтернативный ме­тод измерения магнитных
    моментов шариков}
  \end{figure}
  Повесим на цепочке из  шариков груз и будем постепенно увеличивать его массу, пока цепь не порвётся; взвесим груз на весах. Максимальная масса груза $m_{max} = 257.810 \pm 0.003$ г.
  \item Учитывая, что
  сила в зависимости от расстояния между центрами шаров притяжения убывает как $F\varpropto r^{-4}$, будем учитывать взаимодействие верхнего шара только с 3-4 блийжайшими соседями. Сила сцепления двух одинаковых шаров радиусами $R$ c магнитными моментами $\mathfrak{m}$ равна $$F_0(R) = \frac{6\mk m^2}{(2R)^4} \text{ ед. СГС}$$
  Сила сцепления цепочки равна:  
  $$F = \sum_{i=1}^{4} F_0(iR)= 1.08 F_0(R)\text{ ед. СГС}$$ Тогда из $F = m_{max}g$ получаем: 
   \underline{$\mk m = 69.71 \pm 0.13 \text{ ед. СГС}$}
   \item[0] Как видим, данные для $\mk m$ сильно отличаются (на 50\%), и неправильно будет просто взять среднее. Заметим, что метод А обладает большей \textbf{систематической} погрешностью: на результат сильно влияет тряска рук и начальное (не строго вертикальное) расположение шаров. Поэтому в дальнейшем будем опираться только на это значение.
\end{enumerate}

\newpage
\subsection{Измерение горизонтальной составляющей индукции магнитного
поля Земли}
  \begin{figure}[H]
    \includegraphics*[width=0.4\textwidth]{03_23_12.png}
    \includegraphics*[width=0.4\textwidth]{2023-10-02-17-27-25.png}
    \caption{Крутильные маятники: магнитная стрелка и кольцо}
  \end{figure}
\begin{enumerate}
  \item Соберём магнитное кольцо и убедимся, что упругость
  нити при расчёте периода колебаний можно не учитывать, потому что колебания мало угасают \textit{($\sim1/2$ амплитуды за 3 минуты)}
  \item Соберём магнитную стрелку, снимем зависимость периода колебаний $T$ от количества шариков $n$. Период колебаний такого маятника \scriptsize(формула берётся из уравнения колебаний, предполагая, что магнитный момент адитивен)\normalsize:
  \begin{equation}
    T(n) = 2\pi \sqrt{\frac{mR^2}{3\mk m B_\Vert }} \cdot n\ \ \ \ \ \ \ \ \ \ \ \ \ \ \ (2)
    \label{formula}
  \end{equation}
  \item Проверим линейность зависимости. Таблица:
  \begin{figure}[H]
    \begin{tabular}{|r|r|}
    n & T, с \\
    \midrule
     3 & 1.01 \\
     4 & 1.54 \\
     5 & 1.89 \\
     6 & 2.32 \\
     7 & 2.60 \\
     8 & 2.95 \\
     9 & 3.40 \\
     10 &3.73 \\
    \end{tabular}
    \caption{Зависимость T(n)}
  \end{figure}
  \begin{figure}[H]
    \includegraphics*[width=0.8\textwidth]{T(n).png}
  \end{figure}
  \item Угловой коэффициент: 0.379 $c^{-1}$. Тогда по формуле \ref{formula}:
  \underline{$B_\Vert = 38.52 \pm 1.92$ мкТл (не СГС)} 
\end{enumerate}


\newpage
\subsection{Измерение вертикальной составляющей индукции магнитного поля
Земли. Магнитное наклонение.}
\begin{enumerate}
  \item Изготовьте магнитную "стрелку" из шариков и подвесим её за середину с помощью нити на штативе  (как на рисунке)
  \begin{figure}[H]
    \includegraphics*[width=0.7\textwidth]{2023-10-02-17-36-48.png}
    \caption{Измерение вертикальной составляющей поля и магнитного наклонения}
  \end{figure}
  \item Будем постепенно утяжелять груз, чтобы стрелка приняла горизонтальное положение. Снимем зависимость $m_\text{гр}$ массы груза и $r_\text{гр}$ плеча груза (относительно нити; уравновешиваем всегда на последнем от центра промежутке) от количества шаров в стрелке $n$. Таблица:
  \begin{figure}[H]
    \begin{tabular}{|r|c|r|}
       $n$ & $m_\text{гр}, \text{гр} \pm 0.003$ &$r_\text{гр},$ см$\pm 0.01$\\
      \midrule
       4 & 0.391 & 0.59\\
       6 & 0.267 & 1.19\\
       8 & 0.182 & 1.78\\
      10 & 0.119 & 2.38\\
    \end{tabular}
    \caption{Зависимость $m_\text{гр}(n), r_\text{гр}(n)$}      
  \end{figure}
  \item Формула для $B_\bot$ \scriptsize(из механического равновесия)\normalsize:
  $$\mathcal{M}(n) = m_\text{гр}gr_\text{гр} = n \mk m B_\bot$$
  Построим график $\mathcal{M}(n)$
  \begin{figure}[H]
    \includegraphics*[width=0.7\textwidth]{m(n).png}
    \caption{Зависимость момента сил от n}
  \end{figure}
  $\mk m B_\bot$ -- угловой коэффициент, и он равен $31.32 \pm 0.98$ ед. СГС, тогда: \\ \underline{$B_\bot = 44.9 \pm 
  1.5$ мкТл  (не СГС)}
\end{enumerate}


\newpage
\section*{Вывод}
По полученным данным  расичтаем индукцию, магнитное наклонение и магнитный момент:
$$B = \sqrt{B_\bot^2 + B_\Vert^2} = 59.2 \pm 0.2 \text{мкТл}$$ -- хорошо соотносится с табличными значениями (52 мкТл)
$$\beta = arctg(\frac{B_\Vert}B_\bot) = 50\degree$$ -- сильно отличается от наклонения в долгопрудном ($72\degree$)

Магнитный момент Земли:
$${\mk m}_\oplus=\frac{B\cdot R_\oplus}{\sqrt{3\sin ^2 \phi +1}} = $$

\end{document}
