% !TeX root = main.tex
\documentclass[12pt,a4paper]{article}
% \documentclass[a4paper,14pt, draft]{article}

%%% отключение нумерации сраниц
\pagestyle{empty}
%%% значок в itemize
% \renewcommand{\labelitemi}{$\cdot$}

%%% Работа с русским языком
\usepackage{cmap}					% поиск в PDF
\usepackage{mathtext} 				% русские буквы в формулах
\usepackage[T1, T2A]{fontenc}			% кодировка %Т1 посоветовал чат гпт
\usepackage[utf8]{inputenc}			% кодировка исходного текста
\usepackage[english,russian]{babel}	% локализация и переносы
\usepackage{indentfirst}            % красная строка в первом абзаце
\frenchspacing                      % равные пробелы между словами и предложениями

%%% Дополнительная работа с математикой
\usepackage{amsmath,amsfonts,amssymb,amsthm,mathtools} % пакеты AMS
\usepackage{icomma}                                    % "Умная" запятая

%%% Свои символы и команды
\usepackage{centernot} % центрированное зачеркивание символа
\usepackage{stmaryrd}  % некоторые спецсимволы
\usepackage{dsfont}
\usepackage{amsthm}


\renewcommand{\epsilon}{\ensuremath{\varepsilon}}
\renewcommand{\phi}{\ensuremath{\varphi}}
\renewcommand{\kappa}{\ensuremath{\varkappa}}
\renewcommand{\le}{\ensuremath{\leqslant}}
\renewcommand{\leq}{\ensuremath{\leqslant}}
\renewcommand{\ge}{\ensuremath{\geqslant}}
\renewcommand{\geq}{\ensuremath{\geqslant}}
\renewcommand{\emptyset}{\ensuremath{\varnothing}}

\DeclareMathOperator{\sgn}{sgn}
\DeclareMathOperator{\ke}{Ker}
\DeclareMathOperator{\im}{Im}
\DeclareMathOperator{\re}{Re}

\newcommand{\N}{\mathbb{N}}
\newcommand{\Z}{\mathbb{Z}}
\newcommand{\Q}{\mathbb{Q}}
\newcommand{\R}{\mathbb{R}}
\newcommand{\Cm}{\mathbb{C}}
\newcommand{\F}{\mathbb{F}}
\newcommand{\id}{\mathrm{id}}
\newcommand{\imp}[2]{
	(#1\,\,$\ra$\,\,#2)\,\,
}
\newcommand{\Root}[2]{
	\left\{\!\sqrt[#1]{#2}\right\}
}
\newcommand{\RR}{\R}
\newcommand{\NN}{\N}
\renewcommand{\subseteq}{\subset}
\newcommand{\sub}{\subset}
\newcommand{\sconstr}{\;\vert\;}
\newcommand{\thus}{\implies}

\newcommand{\defeq}{\vcentcolon= }
\newcommand{\defev}{\stackrel{\Delta}{\Longleftrightarrow}}
\newcommand{\deriv}[3][1]{%
	\ifthenelse{#1>1}{%
		\frac{\dlta^{#1} {#2}}{\dlta {#3}^{#1}}
	}{%
		\frac{\dlta {#2}}{\dlta {#3}}
	}%
}

\renewcommand\labelitemi{$\triangleright$}

\let\bs\backslash
\let\lra\Leftrightarrow
\let\ra\Rightarrow
\let\la\Leftarrow
\let\emb\hookrightarrow

%%% Перенос знаков в формулах (по Львовскому)
\newcommand{\hm}[1]{#1\nobreak\discretionary{}{\hbox{$\mathsurround=0pt #1$}}{}}

%%% Работа с картинками
\usepackage{graphicx}    % Для вставки рисунков
\setlength\fboxsep{3pt}  % Отступ рамки \fbox{} от рисунка
\setlength\fboxrule{1pt} % Толщина линий рамки \fbox{}
\usepackage{wrapfig}     % Обтекание рисунков текстом

% \usepackage[inkscapeformat=png]{svg} %% svg

%%% Работа с таблицами
\usepackage{array,tabularx,tabulary,booktabs} % Дополнительная работа с таблицами
\usepackage{longtable}                        % Длинные таблицы
\usepackage{multirow}                         % Слияние строк в таблице

%%% Теоремы
\theoremstyle{plain}
\newtheorem*{theorem}{Теорема}
\newtheorem*{lemma}{Лемма}
\newtheorem*{proposition}{Утверждение}
\newtheorem*{exercise}{Упражнение}
\newtheorem*{problem}{Задача}

\theoremstyle{definition}
\newtheorem*{definition}{Определение}
\newtheorem*{corollary}{Следствие}
\newtheorem*{note}{Замечание}
\newtheorem*{reminder}{Напоминание}
\newtheorem*{example}{Пример}

\theoremstyle{remark}
\newtheorem*{solution}{Решение}

%%% Оформление страницы
\usepackage{extsizes}     % Возможность сделать 14-й шрифт
\usepackage{geometry}     % Простой способ задавать поля
\usepackage{setspace}     % Интерлиньяж
\usepackage{enumitem}     % Настройка окружений itemize и enumerate
\setlist{leftmargin=10pt} % Отступы в itemize и enumerate

\geometry{top=15mm}    % Поля сверху страницы
\geometry{bottom=5mm} % Поля снизу страницы
\geometry{left=10mm}   % Поля слева страницы
\geometry{right=10mm}  % Поля справа страницы

\setlength\parindent{15pt}        % Устанавливает длину красной строки 15pt
\linespread{1}                  % Коэффициент межстрочного интервала
%\setlength{\parskip}{0.5em}      % Вертикальный интервал между абзацами
\setcounter{secnumdepth}{0}      % Отключение нумерации разделов
%\setcounter{section}{-1}         % Нумерация секций с нуля
\usepackage{multicol}			  % Для текста в нескольких колонках
\usepackage{soulutf8}             % Модификаторы начертания
\mathtoolsset{showonlyrefs=true} % показывать номера формул только у тех, у которых есть ссылки по eqref
%%% Содержаниие
% \usepackage{tocloft}
% \tocloftpagestyle{main}
%\setlength{\cftsecnumwidth}{2.3em}
%\renewcommand{\cftsecdotsep}{1}
%\renewcommand{\cftsecpresnum}{\hfill}
%\renewcommand{\cftsecaftersnum}{\quad}

%%% Нумерация уравнений
\makeatletter
\def\eqref{\@ifstar\@eqref\@@eqref}
\def\@eqref#1{\textup{\tagform@{\ref*{#1}}}}
\def\@@eqref#1{\textup{\tagform@{\ref{#1}}}}
\makeatother                      % \eqref* без гиперссылки
\numberwithin{equation}{section}  % Нумерация вида (номер_секции).(номер_уравнения)
\mathtoolsset{showonlyrefs= true} % Номера только у формул с \eqref{} в тексте.

%%% Гиперссылки
\usepackage{hyperref}
\usepackage[usenames,dvipsnames,svgnames,table,rgb]{xcolor}
\hypersetup{
	unicode=true,            % русские буквы в раздела PDF
	colorlinks=true,       	 % Цветные ссылки вместо ссылок в рамках
	linkcolor=black!15!blue, % Внутренние ссылки
	citecolor=green,         % Ссылки на библиографию
	filecolor=magenta,       % Ссылки на файлы
	urlcolor=NavyBlue,       % Ссылки на URL
}

%%% Графика
\usepackage{tikz}        % Графический пакет tikz
\usepackage{tikz-cd}     % Коммутативные диаграммы
\usepackage{tkz-euclide} % Геометрия
\usepackage{stackengine} % Многострочные тексты в картинках
\usetikzlibrary{angles, babel, quotes}
%%% значок в itemize
\renewcommand{\labelitemi}{$\multimap  $}
\title{\texttt{Изучение плазмы газового разряда в неоне \\ 3.5.1}}
\author{}
\date{}

\begin{document}
\maketitle

\textbf{Цель работы:}

\begin{itemize}
  \item  Изучение вольт-амперной характеристики тлеющего раз­ряда
  \item Изучение свойств плазмы методом зондовых характеристик.
\end{itemize}

\textbf{В работе используются:}  
\begin{itemize}
  \item Стеклянная газоразрядная трубка, наполнен­ная неоном
  \item Высоковольтный источник питания
  \item Источник питания посто­янного тока
  \item Делитель напряжения
  \item Потенциометр
  \item Амперметры, вольтметры
  \item Переключатели
\end{itemize}

\section*{Экспериментальная установка}
% \subsection*{\texttt{Важные константы:}}

\begin{figure}[H]
  \includegraphics*[width=\textwidth]{2023-09-02-00-10-53.png}
  \caption{Схема установки для исследования газового разряда}
  \label{fig:ust}
\end{figure}

\noindent Схема экспериментальной установки представлена на рис~\ref{fig:ust}.
  
\section*{Ход работы}
\subsection*{ВAX разряда}
\begin{enumerate}
  \setcounter{enumi}{-1}
  \item Установим переключатель $П_1$ в положение «Анод-I», включим
  мультмиетры $V_1\ и A_1$ в режимы измерение напряжения и силы тока соответственно.
  Плав­но увеличивая выходное напряжение ВИП, определим
  напряжение зажигания разряда $U_\text{заж}$.
  \item С помощью вольтметра $V_1$ и амперметра $A_1$ снимем 
  ВАХ разряда $I_\text{р}(U_\text{р})$ при уменьшениии и увеличении 
  проходящего напряжения.
\end{enumerate}

\subsection*{Зондовые характеристики}
\begin{enumerate}
  \item Уменьшим напряжение ВИП до нуля, переведите переключатель $П_1$
  в положение «Анод-II», $П_2$ в положение «+», 
  включим мультиметры  $V_2\ и A_2$ в режимы 
  измерение напряжения и силы тока соответственно. Плавно увеличим 
  напряжения на ВИП до $\sim5мА$, с помощью потенциометра
  установим напряжение на зонде в $\sim25В$
  \item Снимем ВАХ зонда:\\
  Регулируя ручкой потенциометра напряжение на зонде,
  будем фиксировать показания вольтметра $V_2$ и амперметра $A_2$.
  При приближении к нулю уменьшим шаг, чтобы более точно
  промерять характер ВАХа.
  \item Переведём  $П_2$ в положение «-», и повторим измерения, 
  чтобы снять отрицательныую ветку ВАХа
  \item повторим пункты 2-3 для силы тока на ВИП в $3мА$ и $1.5мА$
\end{enumerate}
\newpage
\subsection*{Обработка результатов}
\begin{enumerate}
  \item Построим ВАХ разряда:
  \begin{figure}[H]
    \includegraphics*[width=\textwidth]{p1.png}
    \caption{ВАХ разряда}
    \label{fig:graph.p1}
  \end{figure}
  Максимальное дифференц% работает как include в C++иальное сопротивление: $R_{дифф} = 0.638\ \Omega$
  \\График соответствует участку графика ЗЖ, сила тока так же убвыает:
  \begin{figure}
    \includegraphics*[width=\textwidth]{2023-09-07-12-28-48.png}
    \caption{Рисунок из лабника}
  \end{figure}
  \item Зондовые характеристики на отдельных графиках в конце.
  \item Оценим температуру электронов по формуле из лабника:
  \begin{tabular}{l|r}
    I, мА & $T_e, K$\\
    \midrule
    5   &  61738 \\
    3   &  52150 \\
    1.5 &  42876 \\
    \end{tabular}
  \item Определим концентрацию электронов по формуле Бома:
  $I_{in} = 0.4n_eeS\sqrt{\frac{2kT_e}{m_i}}$\\
  $n_e = \frac{I_{in}}{0.4eS\sqrt{\frac{2kT_e}{m_i}}}$
  \begin{tabular}{l|r}
    I, мА & $n_e,\cdot 10^{13}$ \\
    \midrule
    5   &  6.2 \\
    3   &  3.5 \\
    1.5 &  1.7 \\
  \end{tabular}
  \item Расчитаем плазменную частоту колебаний:
  \begin{tabular}{l|r}
     & $\omega_p \cdot10^3,$рад/с \\
    \midrule
    5   &  4.51 \\
    3   &  3.38 \\
    1.5 &  2.38 \\
    \end{tabular}
  \item Расчитаем поляризационную длину и длину Дебая:
  \begin{tabular}{lr}
    I, мА & $r_{D_e}, см$ \\
    \midrule
    5 &  201 \\
    3 &  247 \\
    1.5 &  318 \\
    \end{tabular}
     
     \begin{tabular}{lr}
    I, мА & $r_{D}, см$ \\
    \midrule
    5 &  14 \\
    3 &  18 \\
    1.5 &  26 \\
    \end{tabular}
    
\end{enumerate}

\begin{figure}[H]
  \centering
  \includegraphics*[width=0.8\textwidth]{p2.png}
  \caption{Всё на одном листе}
\end{figure}
\begin{figure}[H]
  \centering
  \includegraphics*[width=0.8\textwidth]{отдельный график 1.5мА.png}
  \caption{1.5мА}
\end{figure}
\begin{figure}[H]
  \centering
  \includegraphics*[width=0.8\textwidth]{отдельный график 3мА.png}
  \caption{3мА}
\end{figure}
\begin{figure}[H]
  \centering
  \includegraphics*[width=0.8\textwidth]{отдельный график 5мА.png}
  \caption{5мА}
\end{figure}

\section*{Вывод}
Радиусы не сходятся с табличными значениями, остальное сходится
% \let\clearpage\relax %хуй знает что это но ебаному теху виднее
\end{document}
