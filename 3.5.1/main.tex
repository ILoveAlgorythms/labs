% !TeX root = main.tex
\documentclass[11pt,a4paper]{article}
\input{../preamble.tex}
%%% значок в itemize
\renewcommand{\labelitemi}{$\multimap  $}
\title{\texttt{Изучение плазмы газового разряда в неоне \\ 3.5.1}}
\author{}
\date{}

\begin{document}
\maketitle

\textbf{Цель работы:}
\begin{itemize}
  \item  Изучение вольт-амперной характеристики тлеющего раз­
  ряда
  \item Изучение свойств плазмы методом зондовых характеристик.
\end{itemize}

\textbf{В работе используются:}  
\begin{itemize}
  \item Стеклянная газоразрядная трубка, наполнен­ная неоном
  \item Высоковольтный источник питания
  \item Источник питания посто­янного тока
  \item Делитель напряжения
  \item Потенциометр
  \item Амперметры, вольтметры
  \item Переключатели
\end{itemize}

\section*{Экспериментальная установка}
% \subsection*{\texttt{Важные константы:}}

\begin{figure}[h!]
  \includegraphics*[width=\textwidth]{2023-09-02-00-10-53.png}
  \caption{Схема установки для исследования газового разряда}
  \label{fig:ust}
\end{figure}

\noindent Схема экспериментальной установки представлена на рис~\ref{fig:ust}.
  
\section*{Ход работы}
\subsection*{ВAX разряда}
\begin{enumerate}
  \setcounter{enumi}{-1}
  \item Установим переключатель $П_1$ в положение «Анод-I», включим
  мультмиетры $V_1\ и A_1$ в режимы измерение напряжения и силы тока соответственно.
  Плав­но увеличивая выходное напряжение ВИП, определим
  напряжение зажигания разряда $U_\text{заж}$.
  \item С помощью вольтметра $V_1$ и амперметра $A_1$ снимем 
  ВАХ разряда $I_\text{р}(U_\text{р})$ при уменьшениии и увеличении 
  проходящего напряжения.
\end{enumerate}

\subsection*{Зондовые характеристики}
\begin{enumerate}
  \item Уменьшим напряжение ВИП до нуля, переведите переключатель $П_1$
  в положение «Анод-II», $П_2$ в положение «+», 
  включим мультиметры  $V_2\ и A_2$ в режимы 
  измерение напряжения и силы тока соответственно. Плавно увеличим 
  напряжения на ВИП до $\sim5мА$, с помощью потенциометра
  установим напряжение на зонде в $\sim25В$
  \item Снимем ВАХ зонда:\\
  Регулируя ручкой потенциометра напряжение на зонде,
  будем фиксировать показания вольтметра $V_2$ и амперметра $A_2$.
  При приближении к нулю уменьшим шаг, чтобы более точно
  промерять характер ВАХа.
  \item Переведём  $П_2$ в положение «-», и повторим измерения, 
  чтобы снять отрицательныую ветку ВАХа
  \item повторим пункты 2-3 для силы тока на ВИП в $3мА$ и $1.5мА$
\end{enumerate}

\subsection*{Обработка результатов}
\begin{enumerate}
  \item Построим ВАХ разряда:
  \begin{figure}[h!]
    \includegraphics*[width=\textwidth]{p1.png}
    \caption{ВАХ разряда}
    \label{fig:graph.p1}
  \end{figure}
  Максимальное дифференциальное сопротивление: $R_{дифф} = 0.638\ \Omega$
  \\График соответствует участку графика ЗЖ, сила тока так же убвыает ДОДЕЛАТЬ:
  \begin{figure}[h!]
    \includegraphics*[width=\textwidth]{2023-09-07-12-28-48.png}
    \caption{Рисунок из лабника}
  \end{figure}
  \item Зондовые характеристики на отдельных графиках в конце.
  \begin{figure}[h!]
    \includegraphics*[width=\textwidth]{p2.png}
    \caption{Зондовые зарактеристики}
    \label{fig:graph.p2}
  \end{figure}
  
  
\end{enumerate}

\section*{Вывод}
Энергия активации примерно свопадает с табличным значением.
Как видно из графиков, число Рейнольдса зависит от температуры.
Время релаксации тоже зависит, и как видно, для стеклянных и стальных шариков оно сильно отличается
\end{document}
