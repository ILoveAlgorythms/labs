% !TeX root = main.tex
\documentclass[11pt,a4paper]{article}
\input{../preamble.tex}
%%% значок в itemize
\renewcommand{\labelitemi}{$\multimap  $}
\title{\texttt{Изучение плазмы газового разряда в неоне \\ 3.5.1}}
\author{}
\date{}

\begin{document}
  \maketitle

\textbf{Цель работы:}
\begin{itemize}
  \item  Изучение вольт-амперной характеристики тлеющего раз­
  ряда
  \item Изучение свойств плазмы методом зондовых характеристик.
\end{itemize}

\textbf{В работе используются:}  
\begin{itemize}
  \item Стеклянная газоразрядная трубка, наполнен­ная неоном
  \item Высоковольтный источник питания
  \item Источник питания посто­янного тока
  \item Делитель напряжения
  \item Потенциометр
  \item Амперметры, вольтметры
  \item Переключатели
\end{itemize}

\section*{Экспериментальная установка}
% \subsection*{\texttt{Важные константы:}}

\begin{figure}[h!]
  \includegraphics*[width=\textwidth]{2023-09-02-00-10-53.png}
  \caption{Схема установки для исследования газового разряда}
  \label{fig:ust}
\end{figure}
\noindent Схема экспериментальной установки представлена на рис~\ref{fig:ust}.
  
\section*{Ход работы}
\subsection*{ВAX разряда}
\begin{enumerate}
  \setcounter{enumi}{-1}
  \item Отберём 24 шарика. Для каждого шарика измерим микроскопом
  диаметр в двух положениях и усредним.
  \item Измерим зависимость скорости падения шарика от температуры жидкости.
  Скорость измеряем по формуле $v_\text{уст} = s / t_\text{падения}$
  Для каждой температуры будем проводить измерения по 4-5 раз, фиксируя при этом изменение температуры.
  \item Для каждого из опытов вычислим число Рейнольдса, оценим время и путь релаксации,
  занесём всё в табличку.
  \item График зависимости $ln \eta  (T^{-1})$
  Из МНК, среднеквадратичная ошибка для стекла меньше, чем для стали (0.0011 vs 0.0013), поэтому будем использовать
  значение энергии активации, полученной для стали
  \item Энергия активации:
  \[W = 6.67 * 10 ^ {-20}\frac{\text{Дж}}{\text{Моль}}\]
  \[\sigma_W =  6.14 * 10 ^ {-21}\]
  Приборная погрешность (относительная погрешность человека и линейки -- порядка 1\%, микроскопа -- <<1\%)существенно мала по сравнению со случайной=>
  считаем, что погрешность складывается только из случайной
\end{enumerate}

\subsection*{Зондовые характеристики}


\section*{Графики}

\section*{Вывод}
Энергия активации примерно свопадает с табличным значением.
Как видно из графиков, число Рейнольдса зависит от температуры.
Время релаксации тоже зависит, и как видно, для стеклянных и стальных шариков оно сильно отличается
\end{document}
