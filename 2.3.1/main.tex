% !TeX root = main.tex
\documentclass[11pt,a4paper]{article}
\input{../preamble.tex}
%%% значок в itemize
\renewcommand{\labelitemi}{$\multimap  $}
\title{\texttt{Получение и измерение вакуума \\ 2.3.1}}
\author{}
\date{}

\begin{document}
  \maketitle

\textbf{Цель работы:}
\begin{itemize}
  \item Измерение объёмов форвакуумной и высоковакуумной частей установки
  \item  Определение скорости откачки системы в
  стационарном режиме, а также по ухудшению и по улучшению вакуума.
\end{itemize}

\textbf{В работе используются:} вакуумная установка с манометрами: масляным, термопарным и ионизационным.

\section*{\texttt{Теория}}
По степени разрежения вакуумные установки принято делить на
три класса: 1) низковакуумные — до  $[10^{-2}; 10^{-3}]$ торр; 2) высоковакуумные — $[10^{-4}; 10^{-7}]$ торр; 
3) установки сверхвысокого вакуума — $[10^{-8};10^{-11}]$ торр.
\\ С физической точки зрения низкий вакуум переходит в высокий, когда длина свободного
 пробега молекул газа оказывается сравнима с 
размерами установки (а течение газа становится сугубо молекулярным). 
Cверхвысокий вакуум характерен крайней важностью 
процессов адсорбции и десорбции частиц на поверхности вакуумной камеры.


\section*{Экспериментальная установка}
\subsection*{\texttt{Важные константы:}}
  \begin{itemize}
    \item 
  \end{itemize}

В данной работе изучаются традиционные методы откачки 
механическим форвакуумным насосом до давления 
$10^{-2}$ торр и диффузионным
масляным насосом до давления $10^{-5}$ торр, а также методы измерения
вакуума в этом диапазоне.
\begin{figure}[h!]
  \includegraphics*[width=\textwidth]{ust.png}
  \caption{Схема экспериментальной установки}
  \label{fig:ust}
\end{figure}
\begin{figure}[h!]
  \includegraphics*[width=\textwidth]{ust2.png}
  \caption{Схема действия ротационного двухпластинчатого форвакуумного на-
  соса. В положениях «а» и «б» пластина «А» засасывает разреженный воздух
  из откачиваемого объёма, а пластина «Б» вытесняет ранее захваченный воз-
  дух в атмосферу. В положениях «в» и «г» пластины поменялись ролями}
  \label{fig:ust2}
\end{figure}
\begin{figure}[h!]
  \includegraphics*[width=\textwidth]{ust3.png}
  \caption{Схема работы диффузионного насоса} 
  \label{fig:ust3}
\end{figure}
% \noindent Схема экспериментальной установки представлена на рис~\ref{fig:ust}.

Измеряем пройденное расстояние линейкой, а время -- секундомером,
находим $v_\text{уст}$. Радиус шарика измеряем горизонтальным
компаратором/микроскопом (для каждого шарика измеряем насколько диаметров и берём среднее).
Плотность шариков и жидкости -- табличные значения.\\
Опыты проводятся при нескольких температурах в интервале от
комнатной до $320-330$ К.\\
Для каждой температуры проводим измерения с разными диаметрами шариков.\\
Построим график в координатах $ln\eta (T^{-1})$.\\
Если во всем диапазоне встречающихся в работе 
скоростей и времён релаксации вычисленные по нашей формуле
значения $\eta$ оказываются одинаковыми, то формула Стокса правильно
передаёт зависимость сил от радиуса шарика.
Если всё-таки наблюдается кореляция $\eta$ и $r$, то нужно использовать формулу:
\begin{center}{
  \fbox{
  $\eta = \frac{2}{9}gr^2\frac{\rho - \rho_\text{ж}}{v_\text{уст}} \cdot \frac{1}{1 + 2.4(r/R)}$
}}\end{center}
где R --  радиус сосуда.
  
\section*{Ход работы}
\begin{enumerate}
  \setcounter{enumi}{-1}
  \item шлёп шлёп и померяли
  \item 1-3 по лабнику.
  \item Графики:
  \item Энергия ативации:
  \item Погрешность:
\end{enumerate}

\section*{Вывод}
  \subsection*{\textbf{Графики:}}
  \begin{figure}[h]
    % \includegraphics*[width=\textwidth]{all.png}
    \label{fig:graph1}
  \end{figure}
  \newpage
  \begin{figure}[h]
    % \includegraphics*[width=\textwidth]{dd(1p).png}
    \label{fig:graph2}
  \end{figure}
\end{document}
