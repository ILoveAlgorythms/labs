% !TeX root = main.tex
% ^эта штука нужна, чтобы вс код конимал что ему компилить

% Pro tip: Alt+Z переключает перенос строчек
% чтобы удобно вставлять изображение скачай расширение Paste Image

\documentclass[11pt, a4paper]{article}
\input{../preamble.tex}
\title{Измерение магнитного поля Земли \\ 3.1.3}
\author{Кирилл Нелюбин, Б05-207}
\date{\today}

\begin{document}
\maketitle

\section*{Цель работы:}
исследовать свойства постоянных неодимовых магнитов;
\section*{В работе используются:}  
неодимовые магниты; тонкая нить для изготов­
ления крутильного маятника; медная проволока; электронные весы; секун­
домер; измеритель магнитной индукции; штангенциркуль; брусок, линейка
и штатив из немагнитных материалов; набор гирь и разновесов.

\section*{Экспериментальная установка}
\begin{figure}[H]
  \includegraphics*[width=0.5\textwidth]{2023-10-02-00-43-23.png}
  \caption{Схема установки для исследования газового разряда}
  \label{fig:ust}
\end{figure}
\noindent Схема экспериментальной установки представлена на рис~\ref{fig:ust}.


\section*{Ход работы}
\subsection*{Определение магнитного момента магнитных шариков}
\begin{enumerate}
  \item lalala
  \item[100] так можно отдельным шагам поменять нумерацию
  \setcounter{enumi}{-1}
  \item а так можно сбросить нумерацию
  \item так я обычно пишу что как делаю
\end{enumerate}

\section*{Обработка результатов}\begin{enumerate}
  \item тут тоже хуйня
\end{enumerate}


\section*{Вывод}
\begin{figure}[H]
  
  \begin{tabular}{lr}
    \toprule
    {} &         t \\
  n  &           \\
  \midrule
  3  &  1.008667 \\
  4  &  1.539167 \\
  5  &  1.892667 \\
  6  &  2.316667 \\
  7  &  2.603571 \\
  8  &  2.952000 \\
  9  &  3.399091 \\
  10 &  3.733333 \\
  \bottomrule
\end{tabular}
\caption{табличку нужно ещё чето сделать, либо научиьтся нормально с pd.DataFrame научиться рабоать}
\end{figure}
\end{document}
