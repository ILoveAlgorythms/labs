% !TeX root = main.tex
\documentclass[11pt,a4paper]{article}
\input{../preamble.tex}
%%% значок в itemize
\renewcommand{\labelitemi}{$\multimap  $}
\title{\texttt{Определение энергии активации
по температурной зависимости вязкости
жидкости \\ 2.2.6}}
\author{}
\date{}

\begin{document}
  \maketitle

\textbf{Цель работы:}
\begin{itemize}
  \item  Измерение скорости падения шариков при разной
  температуре жидкости
  \item Вычисление вязкости жидкости по закону
  Стокса и расчёт энергии активации
\end{itemize}

\textbf{В работе используются:}  стеклянный цилиндр с исследуемой
 жидкостью (глицерин); термостат; секундомер; горизонтальный компаратор; 
 микроскоп; мелкие шарики (диаметром около 1 мм)

\section*{\texttt{Теория}}
В жидкостях, как и в кристаллах, каждая
молекула находится в потенциальной яме электрического поля, 
создаваемого окружающими молекулами. Молекулы колеблются со средней
частотой, близкой к частоте колебаний атомов в кристаллических телах
($\backsim 10^{12}$ Гц), с амплитудой, определяемой размерами объёма, 
предоставленного ей соседними молекулами. Глубина потенциальной ямы в
жидкостях больше средней кинетической энергии колеблющейся 
молекулы, поэтому молекулы колеблются вокруг более или 
менее стабильных положений равновесия. Однако у жидкостей различие 
между этими двумя энергиями невелико, так что молекулы нередко выскакивают из
"своей" потенциальной ямы и занимают место в другой.\\
Для перехода в новое состояние молекула должна получить
некоторое количество энергии W, называемую \textit{энергией мотивации}.\\
Температурная зависимость вязкости жидкости 
выражается формулой: \[\eta \backsim e^{W/kT}\]
Экспериментальные исследования показывают, что формула
неплохо работает в небольших температурных интервалах.\\
Формула Стокса для ламинарного обтекания шарика безграничной жидкостью:
\[F = 6\pi\eta rv \]
\fbox{Характер обтекания определяется числом 
Рейнольдса $Re = vr\rho_\text{ж} /\eta$ }(для ламинарного течения Re < 0.5)\\
Время релаксации скорости:
\[\tau = \frac{2}{9} \frac{r^2\rho}{\eta}\]
Тогда $\eta$:

\begin{center}
  \fbox{$\eta = \frac{2}{9}gr^2\frac{\rho - \rho_\text{ж}}{v_\text{уст}}$}
\end{center}

\newpage

\section*{Экспериментальная установка}
\subsection*{\texttt{Важные константы:}}
  \indent Диаметр сосуда:\\
  \indent Длина сосуда:\\
  \indent s = $(10.0 \pm 0.1) \text{см}$

\begin{figure}[h!]
  \includegraphics*[width=\textwidth]{ust.png}
  \caption{Установка для определения коэффициента
  вязкости жидкости}
  \label{fig:ust}
\end{figure}
% \noindent Схема экспериментальной установки представлена на рис~\ref{fig:ust}.

Измеряем пройденное расстояние линейкой, а время -- секундомером,
находим $v_\text{уст}$. Радиус шарика измеряем горизонтальным
компаратором/микроскопом (для каждого шарика измеряем насколько диаметров и берём среднее).
Плотность шариков и жидкости -- табличные значения.\\
Опыты проводятся при нескольких температурах в интервале от
комнатной до $320-330$ К.\\
Для каждой температуры проводим измерения с разными диаметрами шариков.\\
Построим график в координатах $ln(\eta) \bigl(T^{-1}\bigr)$.\\
Если во всем диапазоне встречающихся в работе 
скоростей и времён релаксации вычисленные по нашей формуле
значения $\eta$ оказываются одинаковыми, то формула Стокса правильно
передаёт зависимость сил от радиуса шарика.
Если всё-таки наблюдается кореляция $\eta$ и $r$, то нужно использовать формулу:
\begin{center}{
  \fbox{
  $\eta = \frac{2}{9}gr^2\frac{\rho - \rho_\text{ж}}{v_\text{уст}} \cdot \frac{1}{1 + 2.4(r/R)}$
}}\end{center}
где R --  радиус сосуда.
  
\section*{Ход работы}
\begin{enumerate}
  \setcounter{enumi}{-1}
  \item Отберём 24 шарика. Для каждого шарика измерим микроскопом
  диаметр в двух положениях и усредним.
  \item Измерим зависимость скорости падения шарика от температуры жидкости.
  Скорость измеряем по формуле $v_\text{уст} = s / t_\text{падения}$
  Для каждой температуры будем проводить измерения по 4-5 раз, фиксируя при этом изменение температуры.
  \item Для каждого из опытов вычислим число Рейнольдса, оценим время и путь релаксации,
  занесём всё в табличку.
  \item График зависимости $ln \eta  (T^{-1})$
  \begin{center}
    \includegraphics[width=0.9\textwidth]{graph steel.png}
    \includegraphics[width=0.9\textwidth]{graph glass.png}
  \end{center}
  Из МНК, среднеквадратичная ошибка для стекла меньше, чем для стали (0.0011 vs 0.0013), поэтому будем использовать
  значение энергии активации, полученной для стали
  \item Энергия активации:
  \[W = 6.67 * 10 ^ {-20}\frac{\text{Дж}}{\text{Моль}}\]
  \[\sigma_W =  6.14 * 10 ^ {-21}\]
  Приборная погрешность (относительная погрешность человека и линейки -- порядка 1\%, микроскопа -- <<1\%)существенно мала по сравнению со случайной=>
  считаем, что погрешность складывается только из случайной
\end{enumerate}
\section*{Графики}
\begin{center}
  \includegraphics{re}
  % \caption*{число Рейнольдса}
\end{center}
\begin{center}
  \includegraphics{t}
  % \caption*{время релаксации}
\end{center}
\section*{Вывод}
Энергия активации примерно свопадает с табличным значением.
Как видно из графиков, число Рейнольдса зависит от температуры.
Время релаксации тоже зависит, и как видно, для стеклянных и стальных шариков оно сильно отличается
\end{document}
