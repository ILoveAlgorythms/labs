% !TeX root = main.tex
\documentclass[11pt,a4paper]{article}

\input{preamble.tex}
% \title{$\mathbb{O} \mathbb{B} \mathbb{A} \mathbb{T} $и$ \mathbb{K}$}
\author{\texttt{ремикс кирилла}}\date{}
\begin{document}
\renewcommand{\labelitemi}{$\circ  $}
		\subsection*{Обозначения}
		\begin{itemize}

			\item $\mathbf{S}$  перестановочная
			\item $\mathbf{C}$  циклическая
			\item $\mathbf{D}$  диэдральная (группа симметрий правильного n-угольника)
			\item $\mathbf{Z}$  по сложению
		\end{itemize}
		\subsection*{Вычеты}
		\begin{itemize}
			\item Критерий: $[a]_n$ - обратимый $\Longleftrightarrow a, n -  $ взаимно простые
			\item Группа $\Z^*_n$ - группа обратимых вычетов по модулю n. $|\Z^*_n| = \phi(n)$, $\phi(n) - $ функция Эйлера - количество взаимно простых с данным числом и меньших его.
			\item Теорема Эйлера: НОД $(a, n) = 1 \Rightarrow a^{\phi(n)} \equiv 1 (mod \ n) $
			\item Малая теорема Ферма: $p$ - простое, тогда $a \neq 0 (mod \ p) \Rightarrow a^{p - 1} \equiv 1 \ (mod \ p) $
			\item Группа $\Z_n $ по сложению - остатки по модулю n. $a$ - элемент этой группы, тогда $ord(a) = \frac{\text{НОК(a, n)}}{a} = \frac{n}{\text{НОД(a, n)}} $
		\end{itemize} 
		\subsection*{Свойства групповых операций}
		\begin{itemize}
			\item Закон сокращения: Пусть G - группа $\forall a \in G \ x = y \Leftrightarrow ax = ay \Leftrightarrow xa = ya $
			\item Формула обратного произведения: $(xy)^{-1} = y^{-1}x^{-1} $
		\end{itemize}
		\subsection*{Порядки элементов}
		\begin{itemize}
			\item Пусть  $G$ - группа, тогда $\forall g, h \in G  \ |ghg^{-1}| = |h|$, кроме того, $|gh| = |hg|$.
			\item Пусть $\forall g \in G \  g^2 = e$. Тогда $G$ - абелева.
		\end{itemize}
		\subsection*{Подгруппы и классы смежности}
		\begin{itemize}
			\item Критерий подгруппы. Непустое $H \subset G$ - подгруппа группы $G \Leftrightarrow \forall a, b \in H \ ab^{-1} \in H$
			\item Пересечение подгрупп - подгруппа
			\item Левый смежный класс. Пусть $H < G$ тогда левый смежный класс $gH = \{x \in G : x = gh, h \in H\}$. Тут $g$ фиксировано и отвечает за свой класс смежности. Аналогично определяется правый класс смежности $Hg = \{x \in G: x = hg, h \in H\}$ 
			\item Смежные классы не пересекаются, либо совпадают, то есть смежные классы задают отношение эквивалентности на группе.
			\item Критерий принадлежности классу смежности. Для левого класса: $x, y \in G лежат в одном левом классе смежности по подгруппе H \Leftrightarrow y^{-1}x \in H$.
			
			Для правого класса: $x, y \in G лежат в одном правом классе смежности по подгруппе H \Leftrightarrow xy^{-1} \in H$.
			\item Смежные классы не подгруппы в общем случае!
			\item $\left(G : H\right)$ - индекс группы - количество смежные классов по данной подгруппе H. Во всех классах смежности одинаковое количество элементов. 
		\end{itemize}    
		
		\subsection*{Теорема Лагранжа, ее следствие и применения}
		\begin{itemize}
			\item Теорема Лагранжа (мега-имба). Пусть G - группа, $H < G$. Тогда $|G| = (G:H) \cdot |H|$. То есть количество элементов в группе равно количеству элементов в подгруппе, помноженному на количество смежных классов по этой подгруппе.
			\item Следствие 1: Порядок подгруппы делит порядок группы.
			\item Следствие 2: $|G| = p -\text{простое}$. Тогда в $G$ нет несобственных подгрупп (то есть подгруппы только $\left\{e\right\}, G$)
			\item Следствие 3: Порядок элемента делит порядок группы --- OVERPOWERED
			\item Следствие 4: $|G| = p - \text{простое} \Rightarrow G - \text{абелева}$ (доказывается через группу, порожденную одним элементом)
			\item Следствие 5: Пусть $|G| = n, тогда \forall g \in G: \ g^n = e$ (так как порядок элемента делит порядок группы)
		\end{itemize}
		
		\subsection*{Изоморфизм} Изоморфизм - биекция из одной группы в другую, сохраняющая операцию, то есть $\varphi(xy) = \varphi(x)\varphi(y)$
		\begin{itemize}
			\item Необходимое условие существования изоморфизма - порядки элементов совпадают.
			Проверка на изоморфизм - построить биекцию (можно табличкой)
			\item Способы доказать, что изоморфизма нет: проверить количество элементов (должны быть равны), проверить порядки элементов, проверить на абелевость (сказать, что одна - абелева, а вторая - нет).
		\end{itemize}
	
		\subsection*{Прямое произведение групп} Прямое произведение групп - все выполняется покомпонентно.
		\begin{itemize}
			\item НОД$(p, q) = 1$, тогда $C_p \times C_q \cong C_{pq}$.
			\item $ord(g, h) = \text{НОК}(ord(g), ord(h)) $
		\end{itemize}

		
		\subsection*{ Перестановки} 
		Если перестановка $\pi$ представима в виде произведения циклов длины $l_0, l_1, ..., l_n$, то $ord(\pi) = \text{НОК}(l_0, l_1, ..., l_n)$. \\
		Четность перестановки. Эквивалентные определения:
		\begin{enumerate}
			\item Перестановка содержит четное число инверсий
			\item Перестановка представима в видел произведения четного числа транспозиций
			\item Перестановка представима в виде произведения циклов длины 3
			\item $n + k - \text{четно}$, где $k$ - число циклов (вместе с тривиальными)
		\end{enumerate}
		\begin{itemize}
			\item Любой цикл длины 3 представим в виде 2-х транспозиций.
			\item Цикл длины $n$ представим в виде $n$ транспозиций. $(a_1, a_2, a_3, ..., a_n) = (a_1, a_n) \circ (a_1, a_{n - 1}) \circ ... \circ (a_1, a_2) $
			\item Умножение на транспозицию изменяет количество циклов на $\pm 1$.
			\item Четные перестановки образуют подгруппу $A_n$. $|A_n| = \frac{n!}{2}$
			\item Перестановки сопряжены тогда и только тогда, когда у них совпадают длины циклов.
		\end{itemize}
		
		\subsection*{ Гомоморфизм} Гомоморфизм - отображение, сохраняющее операцию $\varphi: G \rightarrow H$, $\varphi(xy) = \varphi(x) \cdot \varphi(y) $. \\ 
		Ядро $Ker \varphi = \{g \in G:  \varphi(g) = e_H\}  $ Ядро - подгруппа G \\
		Образ $Im \varphi = \left\{h \in H: \exists g \in G \ \varphi(g) = h\right\}$ Образ - подгруппа H. \\
		\begin{itemize}
			\item Композиция гомоморфизмов - гомоморфизм
			\item $\varphi: G \rightarrow H$ - гомоморфизм, тогда $\varphi(e)$ - нейтральный в H, \newline $\varphi(a^{-1}) = \varphi(a)^{-1} $
			\item $|G| = |Im \varphi| \cdot |Ker \varphi|$ (G - группа аргументов гомоморфизма, как в определении выше) - следствие из теоремы Лагранжа.
			\item В обозначениях того же определения: если $K < H$, тогда $\varphi^{-1}(H)$ - подгруппа G.
		\end{itemize}
		
		\subsection*{ Нормальные подгруппы, фактор-группа}
		$H \vartriangleleft G \Leftrightarrow \forall g \in G: gH = Hg$, то есть правый смежный класс совпадает с левым смежным классом по нормальной подгруппе
		\begin{itemize}
			\item Ядро любого гомоморфизма - нормальная подгруппа.
			\item Любая нормальная подгруппа является  ядром некоторого гомоморфизма. 
			Фактор-группа - группа из смежных классов по ядру, то есть смежные классы мы воспринимаем как элементы фактор группы. (свойства классов смежности есть выше).
			\item Любая подгруппа абелевой группа - нормальная
			\item ОСНОВНАЯ ТЕОРЕМА ГОМОМОРФИЗМА: $Im \varphi \cong G/Ker \varphi$ - образ изоморфен фактору группы по ядру. 
			\item Подгруппа индекса 2 всегда нормальна
			Пусть $\psi: G \rightarrow G$ - автоморфизм (то есть изоморфизм в самого себя) и $H$ - нормальная подгруппа $G$. Тогда $\psi(H)$ - нормальна и $G/H \cong G/\psi(H)$
			\item $\Z/n\Z \cong \Z_n$
			
		\end{itemize}
		
			
		\subsection*{Сопряжения}  Пусть G - группа. Тогда элемент $b$ сопряжен с $a$ посредством $g$, если $b = g^{-1}ag$. Сопряжение - отношение эквивалентности.\\
		Образуются классы сопряженности, отвечающие отношению эквивалентности.
		\begin{itemize}
			\item Нормализатор $N_x = \left\{y| yx = xy\right\}, y = x^{-1}yx $, то есть все те $y$, которые сопряжены с собой посредством $x$. Ну либо же все те $y$, которые коммутируют с $x$.
			\item Нормализатор - подгруппа
			\item $|N_x| \cdot |\left[x\right]| = |G|, \ \ \left[x\right] - $ класс сопряженности по x.
		\end{itemize}

		
		\subsection*{ Диэдральная группа} Диэдральная группа $D_n$ - группа симметрий правильного n - угольника.
		\begin{itemize}
			\item Количество элементов: $|D_n| = 2n$
			\item Элементы: id, n-1 поворот, n осевых симметрий: n - нечетно, тогда все осевые симметрии - из вершины к середине противоположной стороны. n - четно, тогда n/2 симметрий по серединам сторон, n/2 симметрий по вершинам.
			
			\item Осевую симметрию можно получить сопряжением одной осевой симметрии поворотом: $x = srs^{-1}$, r - симметрия, s - поворот.
			\item Также $s^{-1} = rsr$
			\item $D_n$ не изоморфна $C_n \times C_2$, так как она не абелева.
		\end{itemize}
		
		\subsection*{ Действия групп} Действие группы $G$ на множестве $X$ - гомоморфизм $G \rightarrow S(x)$, то есть каждому элементу из группы мы сопоставляем некоторую перестановку на множестве. 
		\begin{itemize}
		\item Действие левыми сдвигами (группа действует сама на себя) $g(x): x \mapsto gx$
		\item Действие сопряжением (группа действует сама на себя) $g(x): x \mapsto g^{-1}xg$. Тут $Stab_x = N_x$ и $Orb(x) = \left[x\right] $
		\item Орбита действия: $Orb(x) = \left\{y \in X: \exists g \in G: g(x) = y\right\}$
		\item Стабилизатор: $Stab(x) = \left\{g \ in G: g(x) = x\right\}$ - все элементы G, которые оставляют x на месте. 
		\item Стабилизатор - подгруппа
		\item Орбиты - классы эквивалентности
		\item $|Orb(x)| = (G : Stab_x)$
		\item По теореме Лагранжа $|G| = |Orb(x)| \cdot |Stab_x|$
		
		\item Стабилизаторы элементов одной орбиты сопряжены.
		
		\item Центр группы $Z(G) = \left\{z \in G| \forall g \in G: zg = gz\right\}$ - те элементы, которые коммутируют со всеми. Центр - ядро при действием сопряжением.
		
		\item Если G - p-группа (ее порядок равен $p^n$, p - простое). Тогда центр $Z(G)$ - нетривиален, то есть в нем больше 1 элемента. 			
		
		\item Лемма Бернсайда: $\#\text{орбит} = \frac{1}{|G|}\sum_{g \in G} |X^g|$, где $X^g $ - это точки, неподвижные под действием g. 
		\end{itemize}
		
		
		\subsection*{Группы симметрий правильный многогранников} 
		$Cube \cong S_4$ (доказывается через диагонали) $|Cube| = |Oct| = 24$\\
		$Tetra \cong A_4, |Tetra| = 12$, $Dodec \cong Iso \cong A_5,  |Dodec| = 60$\\
		
		\subsection*{ Fun Facts}
		\begin{itemize}
		\item Если $ p > 2 - \text{простое}$, тогда уравнение $x^2 \equiv 1 \ \mod{p}$ имеет только 2 решения: 1 и -1. (доказывается через $(x - 1)(x + 1) \equiv 0 \mod{p} $)
		\item $\sum_{d|n}\varphi(d) = n$, где $\varphi(d) -$функция Эйлера.
		\item Мультипликативность функции Эйлера: $\text{НОД}(p, q) = 1 \Rightarrow \varphi(pq) = \varphi(p) \cdot \varphi(q)$
		\item Необходимое и достаточное условие  квадратичного вычета: a - кв.вычет по модулю p - простого $\Leftrightarrow a^{\frac{p - 1}{2}} \equiv 1 \mod{p}$ 
		\item Конечно порожденная абелева группа изоморфна прямому произведению циклических
		
		\end{itemize} 
\end{document}