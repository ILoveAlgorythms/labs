% !TeX root = main.tex
\documentclass[14pt,a4paper]{article}
% \documentclass[a4paper,14pt, draft]{article}

%%% отключение нумерации сраниц
\pagestyle{empty}
%%% значок в itemize
% \renewcommand{\labelitemi}{$\cdot$}

%%% Работа с русским языком
\usepackage{cmap}					% поиск в PDF
\usepackage{mathtext} 				% русские буквы в формулах
\usepackage[T1, T2A]{fontenc}			% кодировка %Т1 посоветовал чат гпт
\usepackage[utf8]{inputenc}			% кодировка исходного текста
\usepackage[english,russian]{babel}	% локализация и переносы
\usepackage{indentfirst}            % красная строка в первом абзаце
\frenchspacing                      % равные пробелы между словами и предложениями

%%% Дополнительная работа с математикой
\usepackage{amsmath,amsfonts,amssymb,amsthm,mathtools} % пакеты AMS
\usepackage{icomma}                                    % "Умная" запятая

%%% Свои символы и команды
\usepackage{centernot} % центрированное зачеркивание символа
\usepackage{stmaryrd}  % некоторые спецсимволы
\usepackage{dsfont}
\usepackage{amsthm}


\renewcommand{\epsilon}{\ensuremath{\varepsilon}}
\renewcommand{\phi}{\ensuremath{\varphi}}
\renewcommand{\kappa}{\ensuremath{\varkappa}}
\renewcommand{\le}{\ensuremath{\leqslant}}
\renewcommand{\leq}{\ensuremath{\leqslant}}
\renewcommand{\ge}{\ensuremath{\geqslant}}
\renewcommand{\geq}{\ensuremath{\geqslant}}
\renewcommand{\emptyset}{\ensuremath{\varnothing}}

\DeclareMathOperator{\sgn}{sgn}
\DeclareMathOperator{\ke}{Ker}
\DeclareMathOperator{\im}{Im}
\DeclareMathOperator{\re}{Re}

\newcommand{\N}{\mathbb{N}}
\newcommand{\Z}{\mathbb{Z}}
\newcommand{\Q}{\mathbb{Q}}
\newcommand{\R}{\mathbb{R}}
\newcommand{\Cm}{\mathbb{C}}
\newcommand{\F}{\mathbb{F}}
\newcommand{\id}{\mathrm{id}}
\newcommand{\imp}[2]{
	(#1\,\,$\ra$\,\,#2)\,\,
}
\newcommand{\Root}[2]{
	\left\{\!\sqrt[#1]{#2}\right\}
}
\newcommand{\RR}{\R}
\newcommand{\NN}{\N}
\renewcommand{\subseteq}{\subset}
\newcommand{\sub}{\subset}
\newcommand{\sconstr}{\;\vert\;}
\newcommand{\thus}{\implies}

\newcommand{\defeq}{\vcentcolon= }
\newcommand{\defev}{\stackrel{\Delta}{\Longleftrightarrow}}
\newcommand{\deriv}[3][1]{%
	\ifthenelse{#1>1}{%
		\frac{\dlta^{#1} {#2}}{\dlta {#3}^{#1}}
	}{%
		\frac{\dlta {#2}}{\dlta {#3}}
	}%
}

\renewcommand\labelitemi{$\triangleright$}

\let\bs\backslash
\let\lra\Leftrightarrow
\let\ra\Rightarrow
\let\la\Leftarrow
\let\emb\hookrightarrow

%%% Перенос знаков в формулах (по Львовскому)
\newcommand{\hm}[1]{#1\nobreak\discretionary{}{\hbox{$\mathsurround=0pt #1$}}{}}

%%% Работа с картинками
\usepackage{graphicx}    % Для вставки рисунков
\setlength\fboxsep{3pt}  % Отступ рамки \fbox{} от рисунка
\setlength\fboxrule{1pt} % Толщина линий рамки \fbox{}
\usepackage{wrapfig}     % Обтекание рисунков текстом

% \usepackage[inkscapeformat=png]{svg} %% svg

%%% Работа с таблицами
\usepackage{array,tabularx,tabulary,booktabs} % Дополнительная работа с таблицами
\usepackage{longtable}                        % Длинные таблицы
\usepackage{multirow}                         % Слияние строк в таблице

%%% Теоремы
\theoremstyle{plain}
\newtheorem*{theorem}{Теорема}
\newtheorem*{lemma}{Лемма}
\newtheorem*{proposition}{Утверждение}
\newtheorem*{exercise}{Упражнение}
\newtheorem*{problem}{Задача}

\theoremstyle{definition}
\newtheorem*{definition}{Определение}
\newtheorem*{corollary}{Следствие}
\newtheorem*{note}{Замечание}
\newtheorem*{reminder}{Напоминание}
\newtheorem*{example}{Пример}

\theoremstyle{remark}
\newtheorem*{solution}{Решение}

%%% Оформление страницы
\usepackage{extsizes}     % Возможность сделать 14-й шрифт
\usepackage{geometry}     % Простой способ задавать поля
\usepackage{setspace}     % Интерлиньяж
\usepackage{enumitem}     % Настройка окружений itemize и enumerate
\setlist{leftmargin=10pt} % Отступы в itemize и enumerate

\geometry{top=15mm}    % Поля сверху страницы
\geometry{bottom=5mm} % Поля снизу страницы
\geometry{left=10mm}   % Поля слева страницы
\geometry{right=10mm}  % Поля справа страницы

\setlength\parindent{15pt}        % Устанавливает длину красной строки 15pt
\linespread{1}                  % Коэффициент межстрочного интервала
%\setlength{\parskip}{0.5em}      % Вертикальный интервал между абзацами
\setcounter{secnumdepth}{0}      % Отключение нумерации разделов
%\setcounter{section}{-1}         % Нумерация секций с нуля
\usepackage{multicol}			  % Для текста в нескольких колонках
\usepackage{soulutf8}             % Модификаторы начертания
\mathtoolsset{showonlyrefs=true} % показывать номера формул только у тех, у которых есть ссылки по eqref
%%% Содержаниие
% \usepackage{tocloft}
% \tocloftpagestyle{main}
%\setlength{\cftsecnumwidth}{2.3em}
%\renewcommand{\cftsecdotsep}{1}
%\renewcommand{\cftsecpresnum}{\hfill}
%\renewcommand{\cftsecaftersnum}{\quad}

%%% Нумерация уравнений
\makeatletter
\def\eqref{\@ifstar\@eqref\@@eqref}
\def\@eqref#1{\textup{\tagform@{\ref*{#1}}}}
\def\@@eqref#1{\textup{\tagform@{\ref{#1}}}}
\makeatother                      % \eqref* без гиперссылки
\numberwithin{equation}{section}  % Нумерация вида (номер_секции).(номер_уравнения)
\mathtoolsset{showonlyrefs= true} % Номера только у формул с \eqref{} в тексте.

%%% Гиперссылки
\usepackage{hyperref}
\usepackage[usenames,dvipsnames,svgnames,table,rgb]{xcolor}
\hypersetup{
	unicode=true,            % русские буквы в раздела PDF
	colorlinks=true,       	 % Цветные ссылки вместо ссылок в рамках
	linkcolor=black!15!blue, % Внутренние ссылки
	citecolor=green,         % Ссылки на библиографию
	filecolor=magenta,       % Ссылки на файлы
	urlcolor=NavyBlue,       % Ссылки на URL
}

%%% Графика
\usepackage{tikz}        % Графический пакет tikz
\usepackage{tikz-cd}     % Коммутативные диаграммы
\usepackage{tkz-euclide} % Геометрия
\usepackage{stackengine} % Многострочные тексты в картинках
\usetikzlibrary{angles, babel, quotes}
%%% значок в itemize
\renewcommand{\labelitemi}{$\multimap  $}
\title{\texttt{Петля гистерезиса (динамический метод)\\ 3.4.5}}
\author{\texttt{Кирилл Нелюбин, Б05-207}}
\date{\texttt{\today}}
\begin{document}
\maketitle

% \textbf{Цель работы:}
% \begin{itemize}
%   \item  Изучение вольт-амперной характеристики тлеющего раз­
%   ряда
%   \item Изучение свойств плазмы методом зондовых характеристик.
% \end{itemize}

\textbf{В работе используются:}  
понижающий трансформатор, реостат, резистор, интегрирующая цепочка, амперметр и вольтметр (мультиметры), электронный оcциллогра, делитель напряжения, переключатель, тороидальные образцы с дву-
мя обмотками.


\section*{Экспериментальная установка}
% \subsection*{\texttt{Важные константы:}}

\begin{figure}[H]
  \includegraphics*[width=0.8\textwidth]{2023-09-30-11-06-33.png}
  \caption{Схема установки для исследования намагничевания образцов}
  \label{fig:ust}
\end{figure}
\subsection*{Параметры установки}
\begin{itemize}
  \item $R_\text{и} = 20$ кОм
  \item $C_\text{и} = 20$ мкФ
  \item $R_0 = 0.22$ Ом
\end{itemize}

% \noindent Схема экспериментальной установки представлена на рис~\ref{fig:ust}.
\newpage
\section*{Ход работы}
\subsection*{Петля гистерезиса на экране осцилографа}
\begin{enumerate}
  % \setcounter{enumi}{-1}
  \item Для наблюдения петли гистерезиса на экране осцилографа соберём схему согласно рис~\ref{fig:ust}. (соединительные провода показаны на рисунке стрелками).
  Установим реостат $R_1$ на минимальное выходное напряжение, ключ $K_0$x в положение "П";
  Настроим осциллограф согласно техническому описанию, включим в сеть, подготовим к работе амперметр А.
  \item \label{p:start} С помощью потенциометра $R_1$ подберём ток питания в намагничивающей обмотке так, чтобы на экране наблюдалась предельная петля гистерезиса, отрегулируем масштаб, чтобы петля занимала большую часть экрана, зафиксируем полученное изображение
  \item\label{p:end} Плавно уменьшая ток насыщения, поточечно снимем кривую намагничевания.
  % \item Расчитаем цену деления шкалы ЭО для петли А/м для оси Х по формуле (4.16) Введения:$$H=\frac{IN_0}{2\pi R}$$ (ток $I=K_x/R_0,\ \  R, N_0$ -- параметры установки);\\
  % в теслах на деление для оси $Y$ по формуле: 
  % $$B=\frac{R_\text{и}C_\text{и}}{SN_\text{и}}U_\text{вых}$$ ($U_\text{вых}=K_Y,\ \  R_\text{и}, C_\text{и}$ -- параметры установки)
  \item Повторим пункты \ref{p:start}-\ref{p:end} для всех катушек.
\end{enumerate}

\subsection*{Проверка калибровки осциллографа}
\begin{enumerate}
  \item \textit{Проведём калибровку горизонтальной оси ЭО.} Закоротим обмотку $N_0$, с помощью реостата  $R_1$ подберём такой ток, чтобы горизонтальная прямая занимала большую часть экрана ЭО. По формуле:
  $$K_X=2R_0\sqrt{2}I_\text{ЭФ} / 2x$$
  \footnotesize формула берётся из соображений перевода показаний амперметра в показания ЭО \normalsize\\
  (2x -- длина горизонтальной прямой) расчитаем $K_X$. Повторим измерения для всех $K_X$, которые мы использовали. Сведём результаты в табличку:
  
  \item \textit{Проведём калибровку вертикальной оси ЭО.} Соединим
  вход Y ЭО с клеммами делителя 1/100 и общий (земля), переведём $K_0$ в положение Д. Не меняя рабочего коэициента $K_Y$, подберём с
  помощью реостата $R_1$ напряжение, при котором вертикальная прямая занимает почти весь экран. Измерим длину прямой 2y в см (двойную амплитуду
  сигнала).
  Подключим мультитметр V его к тем же
  точкам делителя и определим эективное значение напряжения.
  Запишите напряжение $U_\text{ЭФ}$, величину сигнала на экране 2y в см и коэфициент усиления осциллограа $K_Y$. По формуле:
  $$K_Y=2\sqrt{2}U_\text{ЭФ} / 2y$$
  \footnotesize формула берётся из соображений перевода показаний амперметра в показания ЭО \normalsize\\
  (2x -- длина горизонтальной прямой) расчитаем $K_X$.
  Повторим измерения для всех $K_Y$, которые мы использовали. Сведём результаты в табличку (см приложение, рис \ref{x}, \ref{y})
\end{enumerate}


\subsection*{Обработка результатов}
\begin{enumerate}
  \item Расчитаем коэфициенты преобразования отклонений по осям ЭО в напряженность $H$ и индукцию $B$ по формулам:
  $$H = \frac{I}{N_0}{2\pi R}, \ \ \ \ \ \ B = \frac{R_\text{и}C_\text{и}}{SN\text{и}}U_\text{вых}$$
  \item Рассчитаем амплитуду
  $H_{max}$, соответствующую состоянию насыщения (предельное поле). Вычислим индукцию насыщения $B_s$.
  \item Рассчитаем коэрцитивное поле $H_c$ и остаочную индукцию $B_r$ для каждого образца. По начальным кривым намагничивания оцените начальные и максимальные
значения диеренциальной магнитной проницаемости $\mu_\text{диф}$ = $dB/dH$.
  Сведём в таблицу:
  \begin{figure}[H]
    \begin{tabular}{l|ccc}
      {} &        Fe-Ni &            Fe &         Fe-Si \\
      \midrule
      N0  &    15 &     45 &     20 \\
      Nu  &   300 &    400 &    200 \\
      $S, \text{см}^2$   &     0.66 &      3.00 &      2.00 \\
      $2\pi R,$ см &     14.10 &      25.00 &      11.00 \\
      I, А   &     0.229 &      0.937 &      1.298 \\
      $K_y,$ мВ  &     50 &      50 &      20 \\
      H, A/м   &  2 436 &  1 6878 &  23 596 \\
      B, Тл   &     1.010 &      0.167 &      0.200 \\
      \end{tabular}
  \end{figure}
\end{enumerate}


\subsection*{Определение характерного времени интегрерующей ячейки}\begin{enumerate}
  \item[0] Формула характерного времени:
  $$\tau = RC = \frac{U_\text{ВХ}}{\omega U_\text{ВЫХ}}$$
  \item Подадим на вход ячейки
  напряжение с обмотки 6,3 B трансформатора ($K_0$ в положении П).
  \item Подключим канал Y ЭО ко входу интегрирующей ячейки, отключим канал X. Установите чувствительность $K_Y \sim$ n V /дел. Подберём с помощью реостата $R_1$ такой ток, при котором вертикальная прямая занимает
  большую часть экрана, и определим входное напряжение на RC-цепочке:
  $$U_\text{ВХ} = 2y \cdot K_Y$$
  Не меняя тока, переключим Y-вход ЭО к интегрирующей ёмкости и
  аналогичным образом определим выходное напряжение $U_\text{ВЫХ}$.
  \item  Рассчитаем на месте постоянную времени по формуле из нулевого пункта:\\
  $U_\text{ВХ} = 2 * 4 * 1$В, \ \  $U_\text{ВЫХ} = 0.055$В => \underline{$\tau = 0.46$с}.\\
  Через $RC:$ $\tau = 0.40$с => отклонение -- 13\%
\end{enumerate}

\section*{Вывод}

\newpage
\section*{Приложение}
\subsection*{Петли гистерезиса на экране ЭО}
\begin{figure}[H]
  \includegraphics*[width=\textwidth]{2023-09-30-18-30-19.png}
  \caption{Премаллой}
\end{figure}

\begin{figure}[H]
  \includegraphics*[width=0.48\textwidth]{2023-09-30-18-31-18.png}
  \includegraphics*[width=0.48\textwidth]{2023-09-30-18-31-37.png}
  \caption{Феррит}
\end{figure}

\begin{figure}[H]
  \includegraphics*[width=\textwidth]{2023-09-30-18-32-30.png}
  \caption{Кремнистое железо}
\end{figure}

\begin{figure}[H]
  \centering
  \begin{tabular}{l|c}
    Шкала &Истиное значение \\
    \midrule
    0.100 &  0.118\\
    0.050 &  0.059\\
    0.020 &  0.023\\
    0.010 &  0.012\\
    0.005 &  0.006\caption{Калибровка X}
    \label{x}
  \end{tabular}
\end{figure}
\begin{figure}[H]
\begin{tabular}{l|c}
  Шкала &Истиное значение\\
  \midrule
  0.02 &  0.019 \\
  0.05 &  0.048 \\
  0.01 &  0.009\caption{Калибровка Y}\label{y}
  \end{tabular}
\end{figure}


\end{document}
