% !TeX root = main.tex
\documentclass[11pt,a4paper]{article}

\input{../preamble.tex}
% \title{$\mathbb{O} \mathbb{B} \mathbb{A} \mathbb{T} $и$ \mathbb{K}$}
\title{\texttt{2.1.3 Определение $C_p /C_v$ по скорости звука в газе}}\date{}\author{}
\begin{document}
\maketitle

\section*{Цель работы:}
  \begin{enumerate}
    \item измерение частоты колебаний и длины волны при
  резонансе звуковых колебаний в газе, заполняющем трубу;
   \item пределение
   показателя адиабаты с помощью уравнения состояния идеального газа.
  \end{enumerate}

\section*{В работе используется:} 
  Звуковой генератор ГЗ; электронный осциллограф ЭО; 
  микрофон; телефон; раздвижная труба; теплоизоли-
  рованная труба, обогреваемая водой из термостата; баллон со сжатым
  углекислым газом; газгольдер.

\section*{Теория}
  Скорость звука в газах: 
  \begin{equation*}
    c = \sqrt{\gamma\frac{RT}{\mu}}
  \end{equation*}
  $\gamma$ -- показатель адиабаты.
  Тогда:
  \begin{equation*}
    \gamma = \frac{\mu}{RT}c^2
  \end{equation*}
  $f$ -- частота звука, $\lambda$ -- длина волны, тогда:
  \begin{equation*}
    c = \lambda f
  \end{equation*}
  Чтобы возникали стоячие волны (резонансы), должно выполняться:
  \begin{equation*}
    L = n\frac{\lambda}{2}
  \end{equation*}
  Для k-ой гармоники (относительно самого низкой частоты, при которой возникает стоячая волна):
  \begin{equation*}
    f_k = f_1 + \frac{c}{2L} \cdot (k - 1)
  \end{equation*}

\section*{Экспериментальная установка}
  \subsection*{Важные константы:}
    Длина камеры во втором эксперименте: (740 $\pm$ 1) мм\\
  Микрофон и телефон присоединены к установке через тонкие резиновые трубки. 
  Такая связь достаточна для возбуждения и обнаружения звуковых колебаний 
  в трубе и в то же время мало возмущает эти колебания: при расчётах оба торца трубы можно считать неподвижными, а
  влиянием соединительных отверстий пренебречь.
  Первая установка (Рис. 1) содержит раздвижную трубу с миллиметровой шкалой. 
  Через патрубок (на рисунке не показан) труба может наполняться воздухом 
  или углекислым газом из газгольдера. На
  этой установке производятся измерения $\gamma$ для воздуха и для $CO_2$ . 
  Вторая установка (Рис. 2) содержит теплоизолированную трубу постоянной
  длины. Воздух в трубе нагревается водой из термостата. Температура
  газа принимается равной температуре воды, омывающей трубу. На этой
  установке измеряется зависимость скорости звука от температуры.

\newpage
\section*{Ход эксперимента}
\begin{enumerate}
  \item Дадим  осциллографу прогреться.
  \item Убедимся, что колебания синусоидальной формы с большой амплитудой.
  \item Продуем трубу. Таблица измерений для разных частот в конце, 
  таблица полученной зависимости $\lambda(f)$
  \begin{table}[h!]
    % \vspace{5pt}
    \begin{center}
    \begin{tabular}{|c|c|c|c|}
    \hline
    f, kHz & $\lambda$ / 2, мм & с, м/c & $\sigma_c^\text{сист}$, м/c\\ \hline
    4.02 & 43.2&  347              &  0.05\\ \hline
    5.32 & 54.6&  686 => 343       &  0.07\\ \hline
    2.69 & 63.5&  341              &  0.07\\ \hline
    1.60 & 108 &  345              &  -\\ \hline
    \end{tabular}
    \caption{$\sigma_c^\text{сист} = c \cdot \sqrt{
      (\frac{\sigma_f}{f})^2 + (\frac{\sigma_\lambda}{\lambda})^2}$}
    \end{center}
    $\sigma_\lambda$ вычисляется по мнк
\end{table}\\
Оценим системную погрешность нулем а случайную -- 6 м/c (по разности максимального и минимального значения).
Наилучшее: \fbox{344} м/c.
  \item Продуем трубу углекислым газом. Аналогичная таблица:
  \begin{table}[h!]
    % \vspace{5pt}
    \begin{center}
    \begin{tabular}{|c|c|c|}
    \hline
    f, kHz & $\lambda$ / 2, мм &  с, м/c     \\ \hline
    2.26   & 66                &  298        \\ \hline
    4.10   & 97                &  795 => 398 \\ \hline
    \end{tabular}
    \caption{при каждой частоте только 2 точки, поэтому расчитать погрешность не представляется возможным}
    \end{center}
  \end{table}\\
  Оценим системную погрешность нулем а случайную -- 50 м/c (по полуразности максимального и минимального значения).
Наилучшее: 348 м/c.

  \item Перейдём на вторую установку. Таблица для постоянной длины:
  \begin{table}[h!]
    % \vspace{5pt}
    \begin{center}
    \begin{tabular}{|c|c|}
    \hline
    t, C & с, м/c \\ \hline
    23.7 & 406  \\ \hline
    30.1 & 345  \\ \hline
    40.1 & 351  \\ \hline
    50.1 & 354  \\ \hline
    \end{tabular}
    \caption{$\sigma_c^\text{сист} = c \cdot \frac{\sigma_\lambda}{\lambda}$}
    \end{center}
    $\sigma_\lambda$ вычисляется по мнк
\end{table}\\
Аналогично оценим системную погрешность нулём, случайную -- 9
 м/с, лучшее значение: \fbox{350} м/c (первую точку откинули).


\end{enumerate}
  \section*{Вывод} Воьзмём среднее между полученными скоростями 
  звука в воздухе $\gamma_\text{воздуха} = 
  \frac{\mu_{air}}{RT}c_{air}^2 = \fbox{1.40}$ -- сходиться с табличными значениями.
  
  $\gamma_{CO_2} = \frac{\mu_{CO_2}}{RT}c_{CO_2}^2 = 2.13$ 
  -- сильно отличается от табличных значений.
  Наибольшее влияние оказала слуачайная погрешность в измерении 
  скорости звука.

  
  
  \newpage
  \begin{center}
    
    \section*{Установки}
    \begin{figure}[h]
      \includegraphics[width=0.85\textwidth]{movingtube.png}
      \caption{Установка для измерения скорости звука
      при помощи раздвижной трубы}
    \end{figure}
    \begin{figure}[h]
      \includegraphics[width=0.85\textwidth]{c ot t.png}
      \caption{Установка для изучения зависимости скорости звука
      от температуры}
    \end{figure}
  \end{center}
\end{document}