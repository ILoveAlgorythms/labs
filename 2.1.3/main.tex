% !TeX root = main.tex
\documentclass[11pt,a4paper]{article}

\input{../preamble.tex}
% \title{$\mathbb{O} \mathbb{B} \mathbb{A} \mathbb{T} $и$ \mathbb{K}$}
\title{\texttt{2.1.3 Определение $C_p /C_v$ по скорости звука в газе}}\date{}\author{}
\begin{document}
\maketitle
  \textbf{Цель работы:} $1)$ измерение частоты колебаний и длины волны при
  резонансе звуковых колебаний в газе, заполняющем трубу; $2$) опреде-
  ление показателя адиабаты с помощью уравнения состояния идеаль-
  ного газа.\\
  \textbf{В работе используется:} Звуковой генератор ГЗ; электронный ос-
  циллограф ЭО; микрофон; телефон; раздвижная труба; теплоизоли-
  рованная труба, обогреваемая водой из термостата; баллон со сжатым
  углекислым газом; газгольдер.

\section*{Теория}
  Скорость звука в газах: 
  \begin{equation*}
    c = \sqrt{\gamma\frac{RT}{\mu}}
  \end{equation*}
  $\gamma$ -- показатель адиабаты.
  Тогда:
  \begin{equation*}
    \gamma = \frac{\mu}{RT}c^2
  \end{equation*}
  $f$ -- частота звука, $\lambda$ -- длина волны, тогда:
  \begin{equation*}
    c = \lambda f
  \end{equation*}
  Чтобы возникали стоячие волны (резонансы), должно выполняться:
  \begin{equation*}
    L = n\frac{\lambda}{2}
  \end{equation*}
  Для k-ой гармоники (относительно самого низкой частоты, при которой возникает стоячая волна):
  \begin{equation*}
    f_k= = f_1 + \frac{c}{2L} \cdot (k - 1)
  \end{equation*}

\section*{Экспериментальная установка}
  Микрофон и телефон присоединены к установке через тонкие резиновые трубки. 
  Такая связь достаточна для возбуждения и обнаружения звуковых колебаний 
  в трубе и в то же время мало возмущает эти колебания: при расчётах оба торца трубы можно считать неподвижными, а
  влиянием соединительных отверстий пренебречь.
  \begin{figure}[h]
    \includegraphics[width=\textwidth]{movingtube.png}
    \caption{Установка для измерения скорости звука
    при помощи раздвижной трубы}
  \end{figure}
  \begin{figure}[h]
    \includegraphics[width=\textwidth]{c ot t.png}
    \caption{Установка для изучения зависимости скорости звука
    от температуры}
    \label{lala}
  \end{figure}
  Первая установка (Рис. 1) содержит раздвижную трубу с миллиметровой шкалой. 
  Через патрубок (на рисунке не показан) труба может наполняться воздухом 
  или углекислым газом из газгольдера. На
  этой установке производятся измерения $\gamma$ для воздуха и для $CO_2$ . 
  Вторая установка (Рис. 2) содержит теплоизолированную трубу постоянной
  длины. Воздух в трубе нагревается водой из термостата. Температура
  газа принимается равной температуре воды, омывающей трубу. На этой
  установке измеряется зависимость скорости звука от температуры.

\section*{Ход эксперимента}
figure~\pageref{lala}
\end{document}