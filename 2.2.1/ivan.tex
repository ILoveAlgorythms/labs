\documentclass[a4paper,12pt]{article} % тип документа

% Поля страниц
\usepackage[left=2.5cm,right=2.5cm,
    top=2cm,bottom=2cm,bindingoffset=0cm]{geometry}
    
%Пакет дял таблиц   
\usepackage{multirow} 
    
%Отступ после заголовка    
\usepackage{indentfirst}


% Рисунки
\usepackage{floatrow,graphicx,calc}
\usepackage{wrapfig}

%%% Работа с картинками
\usepackage{graphicx}  % Для вставки рисунков
\graphicspath{{images/}{images2/}}  % папки с картинками
\setlength\fboxsep{3pt} % Отступ рамки \fbox{} от рисунка
\setlength\fboxrule{1pt} % Толщина линий рамки \fbox{}
\usepackage{wrapfig} % Обтекание рисунков и таблиц текстом

% Создаёем новый разделитель
\DeclareFloatSeparators{mysep}{\hspace{1cm}}

% Ссылки?
\usepackage{hyperref}
\usepackage[rgb]{xcolor}
\hypersetup{				% Гиперссылки
    colorlinks=true,       	% false: ссылки в рамках
	urlcolor=blue          % на URL
}


%  Русский язык
\usepackage[T2A]{fontenc}			% кодировка
\usepackage[utf8]{inputenc}			% кодировка исходного текста
\usepackage[english,russian]{babel}	% локализация и переносы


% Математика
\usepackage{amsmath,amsfonts,amssymb,amsthm,mathtools}

%%% Дополнительная работа с математикой
\usepackage{amsmath,amsfonts,amssymb,amsthm,mathtools} % AMS
\usepackage{icomma} % "Умная" запятая: $0,2$ --- число, $0, 2$ --- перечисление


% Что-то 
\usepackage{wasysym}
\title{Лабораторная работа 1.3.3}
\author{Выполнил Юдин Иван Б05-207}

\begin{document}
  \maketitle

  \section*{Аннотация}
В работе измеряется коэффицент вязкости воздуха при помощи воздуха, движущего в тонких трубках с разными скоростями, имея разные числа Рейнольдса.

\textbf{Цель работы:} экспериментально исследовать свойства течения газов по тонким трубкам при различных числах Рейнольдса; выявить область применимости закона Пуазейля и с его помощью определить коэффициент вязкости воздуха.

\textbf{В работе используются:}  система подачи воздуха (компрессор, поводящие трубки); газовый счетчик барабанного типа; спиртовой микроманометр с регулируемым наклоном; набор трубок различного диаметра с выходами для подсоединения микроманометра; секундомер.

\section*{Теоретические сведения}
Сила вязкого трения согласно закону Ньютона:
\begin{equation}
\tau_{xy} = \eta \frac{\partial \upsilon_x}{\partial y}
\end{equation}
$\eta$ - коэффициент динамической вязкости.
Характер течения может быть ламинарным или турбулентным, определяется числом Рейнольдса:
\begin{equation}
Re = \frac{\rho u a}{\eta}
\end{equation} 
$\rho$ - плотность среды, $u$ - характерная скорость потока, $a$ - характерный размер системы.

Формула Пуазейля:
\begin{equation}
Q = \frac{\pi R^4\Delta P}{8\eta l}, \hspace*{20mm} \bar{u} = \frac{Q}{\pi R^2}
\end{equation}
Длина установления:
\begin{equation}
\label{длина}
l_{уст} \approx 0.2R\cdot Re
\end{equation}
Скоростной напор:
\begin{equation}
\tilde{\psi} = \frac{R}{l}\frac{\Delta P}{\rho \bar{u}^2}
\end{equation}
Из теории размерностей:
\begin{equation}
\frac{\Delta P}{l} = C(Re)\cdot \frac{\rho \bar{u}^2}{R}
\end{equation}
При больших числах Рейнольдса параметры течения жидкости не зависят от коэффициента вязкости, поэтому $C(Re)\mapsto const$, oткуда
\begin{equation}
Q = const \cdot R^{5/2} \sqrt{\frac{\Delta P}{\rho l}}
\end{equation}
\section*{Экспериментальная установка}
Схема экспериментальной установки представлена на рис. Поток воздуха поступает через газовый счетчки в металлические трубки, трубки имеют заглушки и отверстия для подключения микроманометра.  \ref{установка}. 

\section*{Ход работы}

Данные изменрений давления от расхода $\Delta P (Q)$ приведены ниже. 

% \begin{center}
%   Результаты измерений разности давлений от расхода $\Delta P (Q)$.  
% \end{center}
\begin{table}[!htb]
  \begin{minipage}{.5\linewidth}
    \caption{Для трубки $d_1$}
    \centering
      \begin{tabular}{|c|c|}
      \hline
      $Q$, мл/c & $\Delta P$, Па \\ \hline
      06,4 &   9,81 \\ \hline
      17,8 &  29,42 \\ \hline
      31,0 &  49,04 \\ \hline
      44,4 &  68,65 \\ \hline
      57,2 &  88,26 \\ \hline
      69,7 & 107,88 \\ \hline
      84,0 & 127,49 \\ \hline
      85,0 & 133,38 \\ \hline
      92,3 & 147,11 \\ \hline
      98,0 & 166,72 \\ \hline
      102,7 & 186,33 \\ \hline
      104,2 & 205,95 \\ \hline
      106,4 & 225,56 \\ \hline
      109,9 & 245,18 \\ \hline
      \end{tabular}
  \end{minipage}
  \begin{minipage}{.5\linewidth}
    \centering
      \begin{tabular}{|c|c|}
      \hline
      $Q$, мл/c &  $\Delta P$, Па \\ \hline
        27,0 &          9,81 \\ \hline
        86,0 &         29,42 \\ \hline
        52,0 &         19,61 \\ \hline
        116,8 &         39,23 \\ \hline
        138,5 &         49,04 \\ \hline
        149,7 &         58,84 \\ \hline
        158,7 &         68,65 \\ \hline
        165,0 &         78,46 \\ \hline
        171,0 &         88,26 \\ \hline
        203,0 &        127,49 \\ \hline
        233,0 &        166,72 \\ \hline
        263,0 &        205,95 \\ \hline
        285,0 &        239,29 \\ \hline
      \end{tabular}
  \end{minipage} 
\end{table}

Ниже на графике крестами выделены точки, начиная с которых начинается турбулетное течение. С помощью МНК построим линейную зависимость для точек, которые предположительно принадлежат ламинарному течению (график и его анализ представлены ниже).

Из формулы Пуазейля имеем: 
% \[\eta = \frac{k \pi R^4}{8 l}\]
% \[Q = \frac{\pi R^4 \cdot \Delta P}{8 \eta l}\]
\[\Delta P = Q \cdot \frac{8 \eta l}{\pi R ^4} = Q \cdot k\]

Тогда для погрешности систематической получаем (с учётом, что погрешность $\Delta P$ есть погрешность измерения высоты столба воды, то есть равна $\sigma_{\Delta P} = 0,2 \cdot 9,807 \cdot  \sigma_h = 1,9$ Па): 
\[\sigma_k^{\text{сист}} = k \cdot \sqrt{\left(\frac{\sigma_{\Delta P}}{{\Delta P}_{\max}}\right)^2 + \left(\frac{\sigma_Q}{Q_{\max}} \right)^2}\]


Из МНК имеем, что:

\begin{gather}
  k = \frac{\langle Q \cdot \Delta P \rangle - \langle Q \rangle \cdot \langle \Delta P \rangle}{\langle {Q}^2 \rangle - {\langle Q \rangle}^2} \\
  \sigma_k^{\text{случ}} = \sqrt{\frac{1}{N} \cdot \left(\frac{\langle {\Delta P}^2 \rangle}{\langle Q^2 \rangle} - k^2 \right)} \\
  \sigma_k = \sqrt{(\sigma_k^{\text{случ}})^2 + (\sigma_k^{\text{сист}})^2} \\
  \sigma_\eta = \eta \sqrt{{\left(\frac{\sigma_k}{k}\right)}^2 + 4^2 \cdot {\left(\frac{\sigma_R}{R}\right)}^2 + {\left(\frac{\sigma_l}{l}\right)}^2}
\end{gather}

% \[k = \frac{\langle Q \cdot \Delta P \rangle - \langle Q \rangle \cdot \langle \Delta P \rangle}{\langle {Q}^2 \rangle - {\langle Q \rangle}^2}\]
\begin{table}
  \caption{Результаты из МНК и графика}
  \centering
    \begin{tabular}{|c|c|c|}
      \hline
      & $d_1$ & $d_2$ \\ \hline
      $k \cdot 10^6 $, $\text{Па $\cdot$ с / м $^3$ }$ & 1,53 & 0,37 \\ \hline
      $\sigma_k^{\text{случ}} \cdot 10^6$, $\text{Па $\cdot$ с / м $^3$ }$ & 0,02 & 0,03 \\ \hline
      $\sigma_k^{\text{сист}} \cdot 10^6$, $\text{Па $\cdot$ с / м $^3$ }$ & 0,02 & 0,01 \\ \hline
      $\sigma_k \cdot 10^6$, $\text{Па $\cdot$ с / м $^3$ }$ & 0,03 & 0,03 \\ \hline
      $\eta \cdot 10^-6$, Па $\cdot$ с & 6,96 & 8,05 \\ \hline
      $\sigma_\eta \cdot 10^-6 $, Па $\cdot$ с & 0,38 & 0,71 \\ \hline 
    \end{tabular}
\end{table}

То есть окончательно имеем, что: 
\begin{align*}
  \eta_1 = (6,96 \pm 0,38) \cdot 10^{-6} \text{Па $\cdot $ с} \\
  \eta_2 = (8,05 \pm 0,71) \cdot 10^{-6} \text{Па $\cdot $ с}
\end{align*}

Далее найдём значение $Re = \frac{\rho Q}{\pi R \eta}$, $\sigma_{Re} = Re \cdot \sqrt{{\left( \frac{\sigma_Q}{Q} \right)} ^ 2 + {\left( \frac{\sigma_R}{R} \right)} ^ 2 + {\left( \frac{\sigma_\eta}{\eta} \right)} ^ 2}$ для каждой трубки.

\begin{itemize}
  \item $d_1$: $Q_\text{крит} = 0,085$ л/ с. То есть $Re = 2304 \pm 140$. 
  \item $d_2$: $Q_\text{крит} = 0,149$ л/ с. То есть $Re = 2370 \pm 208$.
\end{itemize}


Далее определим длину участка трубы, на котором происходит уставновление потока. Построим график $P(x)$ для каждой трубы. Будем проводить прямую через все точки кроме первой, считая, что на них уже должно было почти установиться течение.

\newpage
\begin{itemize}
  \item $d_1$. Ожидаемая длина: $l_{\text{уст}} = 0,2 \cdot R_1 \cdot Re = 0,91$ м.
  \item $d_2$. Ожидаемая длина: $l_{\text{уст}} = 0,2 \cdot R_2 \cdot Re = 1,39$ м.
\end{itemize}

\section*{Вывод}

Экспериментально исследовались свойства течения газов по тонким трубкам при различ-
ных числах Рейнольдса; выявить область применимости закона Пуазейля и с его помощью
определить коэффициент вязкости воздуха. Получили вязкость воздуха:
\begin{align*}
  \eta_1 = (6,96 \pm 0,38) \cdot 10^{-6} \text{Па $\cdot $ с} \\
  \eta_2 = (8,05 \pm 0,71) \cdot 10^{-6} \text{Па $\cdot $ с}
\end{align*}
Учтя, что $\eta_\text{табл} = (1,3) \cdot 10^{-6} $Па $\cdot $ с. Это значение отличается от табличного, возможно это связанно с тем, что был не правильно выбран "нулевой" уровень для измерения давления.

Число Рейнольдса получилось $Re_1 = 2304 \pm 140$ и $Re_2 = 2370 \pm 208$, что тоже отличается от $Re \sim 1000$, что скорее всего также связанно с неправильным выбором "нулевого" уровня.


%   \caption{Зависимость $\Delta P(Q)$.}
%   \caption{Зависимость $P(x)$.}



\end{document}

