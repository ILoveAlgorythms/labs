% !TeX root = main.tex
\documentclass[11pt,a4paper]{article}

% \documentclass[a4paper,14pt, draft]{article}

%%% отключение нумерации сраниц
\pagestyle{empty}
%%% значок в itemize
% \renewcommand{\labelitemi}{$\cdot$}

%%% Работа с русским языком
\usepackage{cmap}					% поиск в PDF
\usepackage{mathtext} 				% русские буквы в формулах
\usepackage[T1, T2A]{fontenc}			% кодировка %Т1 посоветовал чат гпт
\usepackage[utf8]{inputenc}			% кодировка исходного текста
\usepackage[english,russian]{babel}	% локализация и переносы
\usepackage{indentfirst}            % красная строка в первом абзаце
\frenchspacing                      % равные пробелы между словами и предложениями

%%% Дополнительная работа с математикой
\usepackage{amsmath,amsfonts,amssymb,amsthm,mathtools} % пакеты AMS
\usepackage{icomma}                                    % "Умная" запятая

%%% Свои символы и команды
\usepackage{centernot} % центрированное зачеркивание символа
\usepackage{stmaryrd}  % некоторые спецсимволы
\usepackage{dsfont}
\usepackage{amsthm}


\renewcommand{\epsilon}{\ensuremath{\varepsilon}}
\renewcommand{\phi}{\ensuremath{\varphi}}
\renewcommand{\kappa}{\ensuremath{\varkappa}}
\renewcommand{\le}{\ensuremath{\leqslant}}
\renewcommand{\leq}{\ensuremath{\leqslant}}
\renewcommand{\ge}{\ensuremath{\geqslant}}
\renewcommand{\geq}{\ensuremath{\geqslant}}
\renewcommand{\emptyset}{\ensuremath{\varnothing}}

\DeclareMathOperator{\sgn}{sgn}
\DeclareMathOperator{\ke}{Ker}
\DeclareMathOperator{\im}{Im}
\DeclareMathOperator{\re}{Re}

\newcommand{\N}{\mathbb{N}}
\newcommand{\Z}{\mathbb{Z}}
\newcommand{\Q}{\mathbb{Q}}
\newcommand{\R}{\mathbb{R}}
\newcommand{\Cm}{\mathbb{C}}
\newcommand{\F}{\mathbb{F}}
\newcommand{\id}{\mathrm{id}}
\newcommand{\imp}[2]{
	(#1\,\,$\ra$\,\,#2)\,\,
}
\newcommand{\Root}[2]{
	\left\{\!\sqrt[#1]{#2}\right\}
}
\newcommand{\RR}{\R}
\newcommand{\NN}{\N}
\renewcommand{\subseteq}{\subset}
\newcommand{\sub}{\subset}
\newcommand{\sconstr}{\;\vert\;}
\newcommand{\thus}{\implies}

\newcommand{\defeq}{\vcentcolon= }
\newcommand{\defev}{\stackrel{\Delta}{\Longleftrightarrow}}
\newcommand{\deriv}[3][1]{%
	\ifthenelse{#1>1}{%
		\frac{\dlta^{#1} {#2}}{\dlta {#3}^{#1}}
	}{%
		\frac{\dlta {#2}}{\dlta {#3}}
	}%
}

\renewcommand\labelitemi{$\triangleright$}

\let\bs\backslash
\let\lra\Leftrightarrow
\let\ra\Rightarrow
\let\la\Leftarrow
\let\emb\hookrightarrow

%%% Перенос знаков в формулах (по Львовскому)
\newcommand{\hm}[1]{#1\nobreak\discretionary{}{\hbox{$\mathsurround=0pt #1$}}{}}

%%% Работа с картинками
\usepackage{graphicx}    % Для вставки рисунков
\setlength\fboxsep{3pt}  % Отступ рамки \fbox{} от рисунка
\setlength\fboxrule{1pt} % Толщина линий рамки \fbox{}
\usepackage{wrapfig}     % Обтекание рисунков текстом

% \usepackage[inkscapeformat=png]{svg} %% svg

%%% Работа с таблицами
\usepackage{array,tabularx,tabulary,booktabs} % Дополнительная работа с таблицами
\usepackage{longtable}                        % Длинные таблицы
\usepackage{multirow}                         % Слияние строк в таблице

%%% Теоремы
\theoremstyle{plain}
\newtheorem*{theorem}{Теорема}
\newtheorem*{lemma}{Лемма}
\newtheorem*{proposition}{Утверждение}
\newtheorem*{exercise}{Упражнение}
\newtheorem*{problem}{Задача}

\theoremstyle{definition}
\newtheorem*{definition}{Определение}
\newtheorem*{corollary}{Следствие}
\newtheorem*{note}{Замечание}
\newtheorem*{reminder}{Напоминание}
\newtheorem*{example}{Пример}

\theoremstyle{remark}
\newtheorem*{solution}{Решение}

%%% Оформление страницы
\usepackage{extsizes}     % Возможность сделать 14-й шрифт
\usepackage{geometry}     % Простой способ задавать поля
\usepackage{setspace}     % Интерлиньяж
\usepackage{enumitem}     % Настройка окружений itemize и enumerate
\setlist{leftmargin=10pt} % Отступы в itemize и enumerate

\geometry{top=15mm}    % Поля сверху страницы
\geometry{bottom=5mm} % Поля снизу страницы
\geometry{left=10mm}   % Поля слева страницы
\geometry{right=10mm}  % Поля справа страницы

\setlength\parindent{15pt}        % Устанавливает длину красной строки 15pt
\linespread{1}                  % Коэффициент межстрочного интервала
%\setlength{\parskip}{0.5em}      % Вертикальный интервал между абзацами
\setcounter{secnumdepth}{0}      % Отключение нумерации разделов
%\setcounter{section}{-1}         % Нумерация секций с нуля
\usepackage{multicol}			  % Для текста в нескольких колонках
\usepackage{soulutf8}             % Модификаторы начертания
\mathtoolsset{showonlyrefs=true} % показывать номера формул только у тех, у которых есть ссылки по eqref
%%% Содержаниие
% \usepackage{tocloft}
% \tocloftpagestyle{main}
%\setlength{\cftsecnumwidth}{2.3em}
%\renewcommand{\cftsecdotsep}{1}
%\renewcommand{\cftsecpresnum}{\hfill}
%\renewcommand{\cftsecaftersnum}{\quad}

%%% Нумерация уравнений
\makeatletter
\def\eqref{\@ifstar\@eqref\@@eqref}
\def\@eqref#1{\textup{\tagform@{\ref*{#1}}}}
\def\@@eqref#1{\textup{\tagform@{\ref{#1}}}}
\makeatother                      % \eqref* без гиперссылки
\numberwithin{equation}{section}  % Нумерация вида (номер_секции).(номер_уравнения)
\mathtoolsset{showonlyrefs= true} % Номера только у формул с \eqref{} в тексте.

%%% Гиперссылки
\usepackage{hyperref}
\usepackage[usenames,dvipsnames,svgnames,table,rgb]{xcolor}
\hypersetup{
	unicode=true,            % русские буквы в раздела PDF
	colorlinks=true,       	 % Цветные ссылки вместо ссылок в рамках
	linkcolor=black!15!blue, % Внутренние ссылки
	citecolor=green,         % Ссылки на библиографию
	filecolor=magenta,       % Ссылки на файлы
	urlcolor=NavyBlue,       % Ссылки на URL
}

%%% Графика
\usepackage{tikz}        % Графический пакет tikz
\usepackage{tikz-cd}     % Коммутативные диаграммы
\usepackage{tkz-euclide} % Геометрия
\usepackage{stackengine} % Многострочные тексты в картинках
\usetikzlibrary{angles, babel, quotes}
\title{\texttt{Исследование взаимной диффузии газов \\ 2.2.1}}
\author{}
\date{}

\begin{document}
  \maketitle

\textbf{Цель работы:}
\begin{itemize}
  \item  Регистрация зависимости концентрации гелия в воздухе
  от времени с помощью датчиков теплопроводности при разных
  начальных давлениях смеси газов;
  \item Определение коэффициента диффузии
   по результатам измерений.
\end{itemize}
\textbf{В работе используются:}  измерительная установка; форвакуумный
насос; баллон с газом (гелий); манометр; источник питания; магазин
сопротивлений; гальванометр; секундомер.

\section*{\texttt{Теория}}
Плотность потока вещества любого компонента в результате взаимной диффузии
\begin{equation}
  j_a = -D_{ab}\frac{\partial \eta_a}{\partial x},
  j_b = -D_{ba}\frac{\partial \eta_b}{\partial x}
\end{equation}
где $D_{ab} = D_{ba}$ — коэффициент взаимной диффузии компонентов,
$j_a, j_b$ — плотности потока частиц соответствующего сорта. Тогда: \fbox{$J = DS\frac{n_1 - n_2}{l}$}\\
Из закона сохранения количества вещества получаем: 
\begin{equation}
  \Delta n = \Delta n_0 \cdot e^{-t/\tau}
\end{equation}, где $\tau = \frac{V_1V_2}{V_1 + V_2} \frac{l}{SD}$
Для проверки применимости квазистационарного приближения необходимо убедиться,
что время $\tau$ много больше характерного времени
диффузии одной частицы вдоль трубки длиной l: $t_\text{дифф} \sim \frac{l^2}{D} \ll \tau$\\

Приведём теоретическую оценку для коэффициента диффузии. В работе концентрация гелия,
как правило, мала ($n_{He} << n_\text{воздуха}$). Кроме того, атомы гелия легче молекул, составляющих воздух
($m_{He} << m_{N_2}, m_{O_2}$), значит их средняя тепловая скорость велика по сравнению с остальными частицами.
Поэтому перемешивание газов в работе можно приближенно описывать как диффузию
примеси лёгких частиц He на практически стационарном фоне воздуха. Коэффициент диффузии
в таком приближении равен
\[D = \frac{1}{3}\lambda\langle v \rangle\]
где $\lambda = \frac{1}{n\sigma}$ -- длина свободного пробега диффундирующих частиц, $\langle v \rangle = \sqrt{\frac{8kT}{\pi m}}$ -- средняя тепловая скорость.

\section*{Экспериментальная установка}
Важные константы:\\
  \[V_1 = V_2 = (800 \pm 5)\text{ см}^3\]
  \[ \frac{l}{S} = (15,0 \pm 0,1)\text{ см}^{-1}\]
  \[\text{Атомсферное давление: 732 торр}\]
\begin{figure}[h]
  \includegraphics*[width=\textwidth]{ustanovka.png}
  \caption{Установка для исследования взаимной диффузии газов}
  \label{fig:ust}
\end{figure}

Схема экспериментальной установки представлена на рис~\ref{fig:ust}.
Поток воздуха поступает через газовый счетчки в металлические трубки,
трубки имеют заглушки и отверстия для подключения микроманометра.

Принцип работы дачтика теплопроводности:
\begin{figure}
  \includegraphics*[width=\textwidth]{ust2.png}
  \caption{Мостовая схема
  с датчиками теплопроводности для измерения разности концентраций газов}
  \label{fig:ust2}
\end{figure}
Количество тепла, передающееся стенке в единицу времени:
\begin{equation}
  Q = \chi \frac{2 \pi l}{ln(R_\text{}P/r_\text{})}(T_1 - T_2)
\end{equation}
При достаточно малых изменениях концентраций можно ожидать,
что величина тока, проходящего через гальванометр G,
будет пропорциональна разности концентраций (первый
член разложения функции в ряд Тейлора). Эксперименты показывают,
что при разности концентраций, равной 15\%, поправка к линейному за-
кону не превышает 0,5\%, что для наших целей вполне достаточно => \fbox{$N = N_0e^-t/\tau$}, где
N -- показания гальванометра, $N_0$ -- показания в начальный момент времени.
\section*{Ход работы}

\begin{enumerate}
  \item пункты 1-6 -- подготовка установки, выполняем их
  \setcounter{enumi}{6}
  \item Провели измерения для рабочих давлений в 40, 80, 120, 160, 200, 240 торр
  \item Мы получили зависимости U(t) для 6 давлений. $ln(U) = ln(U_0e^{-t/\tau}) = ln(U_0) - \frac{t}{\tau}$ =>
  $k = -1/\tau$ -- угловой коэффициент графика $ln(U)(t)$. 
  \begin{equation}
    -1/\tau = -\frac{V_1 + V_2}{V_1V_2}\frac{SD}{l}\ =>\ 
    \underline{D = \frac{-kl}{S} \frac{V_1V_2}{V_1 + V_2}}
  \end{equation} 
  По формуле МНК:
  \begin{gather}
    \ln \left(\frac{U}{U_0}\right) = -kt \\
    k = -\frac{\langle \ln (U/U_0 \cdot t)  \rangle - \langle \ln (U/U_0) \rangle \cdot \langle t \rangle}{\langle t^2 \rangle - {\langle t \rangle}^2} \\
    \sigma_k^{\text{случ}} = \sqrt{\frac{1}{N} \cdot \left(\frac{ \langle {\ln (U/U_0)}^2 \rangle}{\langle t^2 \rangle} - k^2 \right)}
  \end{gather}
  $\sigma_k^\text{полное} = \sigma_k^\text{случайное}$, т.к. $\sigma_k^\text{систематическое} \sim 0$
  (т.к. оно складывается из измерения U и t, а это считает компьютер)
  
  Погрешность коэффициента диффузии:
  \begin{gather}
    \sigma_D = D \cdot \sqrt{\left(\frac{\sigma_k}{k}\right)^2 + \left(\frac{\sigma_{L/S}}{L/S}\right)^2 + \left(\frac{\sigma_{V_1}}{V_1}\right)^2  + \left(\frac{\sigma_{V_2}}{V_2}\right)^2 + \left(\frac{\sigma_{V_1 + V_2}}{V_1 + V_2}\right)^2 }
  \end{gather}


  занесём в таблицу:
  \begin{table}[h!]
    % \vspace{5pt}
    \begin{center}
    \begin{tabular}{|c|c|c|}
    \hline
    $P_\text{раб}$, торр & $D$, см$^2/$с & $\sigma_D$, см$^2/$с \\ \hline
    40  & 29.7 & 0.38 \\ \hline
    80  & 14.7 & 0.19 \\ \hline
    120 & 10.5 & 0.13 \\ \hline
    160 & 8.4  & 0.11 \\ \hline
    200 & 6.6  & 0.08 \\ \hline
    240 & 5.3  & 0.07 \\ \hline
    \end{tabular}

    \caption{Зависимость коэффициента диффузии от давления}
    \label{data}
    \end{center}
\end{table}




В формуле, связывающей коэффициент диффузии и
длину свободного пробега, подставим $\lambda\langle v \rangle$:
\begin{gather}
  D = \frac{kT}{3\sigma} \sqrt{\frac{8RT}{\pi \mu_{He}}} \frac{1}{P} := C \cdot \frac{1}{P} \\
  C = \frac{\langle D \cdot 1/P \rangle - \langle D \rangle \cdot \langle 1/P \rangle }{\langle 1/P^2 \rangle - {\langle 1/P \rangle} ^ 2} \\
  \sigma_C^{\text{случ}} = \sqrt{\frac{1}{N} \cdot \left(\frac{ \langle {D}^2 \rangle}{\langle 1/P^2 \rangle} - k^2 \right)}
\end{gather}

Построим график зависимости $D$ от $1/P$ чтобы проверить это, учтя, что $\sigma_{1/P} = \sigma_P / P^2$, где $\sigma_P = 7.5$ торр. Подставим значение атмосферного давления и найдём коэффицент диффузии.
\[\frac{kT}{3\sigma} \sqrt{\frac{8RT}{\pi \mu_{He}}} = (1153.6 \pm 19.2) \ \text{ торр*см$^2$/с }\]
\[P_{\text{атм}} = 732 \text{торр}\]\[  D_{\text{атм}} = (2.33 \pm 0,04) \  \text{см$^2$/с}\]

Вспомниая выражения для длины свободного пробега:
\[\lambda = \frac{3D}{\langle \upsilon \rangle} = \frac{3D}{\sqrt{\frac{8RT}{\pi \mu_{He}}}} = (704 \pm 7) \ \text{нм}\]


Подставляем комнатную температуру $T = 295K$ и получаем значение эффективного сечения:

\begin{equation}
    \sigma = \frac{kT}{\sqrt{\lambda P}} = (5.3 \pm 0.3) \cdot 10^{-2} \text{нм$^2$}.
\end{equation}
\end{enumerate}

\section*{Вывод}
Полученная величина коэффициента диффузии отличается от табличной (0.9 торр*см$^2$/с) в 2.5 раза, что
что совпадает только по порядку. Длина свободного пробега отличается в 4 раза, совпадает только по порядку.
Методом исключения полагаю, что наибольшуюю погрешность вносит измерение зависимости кожфицента диффузии от концентрации,
а именно выбранная модель.
\section*{\textbf{Графики:}}
\begin{figure}[h]
  \includegraphics*[width=\textwidth]{all.png}
  \label{fig:ust}
\end{figure}
\newpage
\begin{figure}[h]
  \includegraphics*[width=\textwidth]{dd(1p).png}
  \label{fig:ust}
\end{figure}


\end{document}

