
% \documentclass[a4paper,14pt, draft]{article}

%%% отключение нумерации сраниц
\pagestyle{empty}
%%% значок в itemize
% \renewcommand{\labelitemi}{$\cdot$}

%%% Работа с русским языком
\usepackage{cmap}					% поиск в PDF
\usepackage{mathtext} 				% русские буквы в формулах
\usepackage[T1, T2A]{fontenc}			% кодировка %Т1 посоветовал чат гпт
\usepackage[utf8]{inputenc}			% кодировка исходного текста
\usepackage[english,russian]{babel}	% локализация и переносы
\usepackage{indentfirst}            % красная строка в первом абзаце
\frenchspacing                      % равные пробелы между словами и предложениями

%%% Дополнительная работа с математикой
\usepackage{amsmath,amsfonts,amssymb,amsthm,mathtools} % пакеты AMS
\usepackage{icomma}                                    % "Умная" запятая

%%% Свои символы и команды
\usepackage{centernot} % центрированное зачеркивание символа
\usepackage{stmaryrd}  % некоторые спецсимволы
\usepackage{dsfont}
\usepackage{amsthm}


\renewcommand{\epsilon}{\ensuremath{\varepsilon}}
\renewcommand{\phi}{\ensuremath{\varphi}}
\renewcommand{\kappa}{\ensuremath{\varkappa}}
\renewcommand{\le}{\ensuremath{\leqslant}}
\renewcommand{\leq}{\ensuremath{\leqslant}}
\renewcommand{\ge}{\ensuremath{\geqslant}}
\renewcommand{\geq}{\ensuremath{\geqslant}}
\renewcommand{\emptyset}{\ensuremath{\varnothing}}

\DeclareMathOperator{\sgn}{sgn}
\DeclareMathOperator{\ke}{Ker}
\DeclareMathOperator{\im}{Im}
\DeclareMathOperator{\re}{Re}

\newcommand{\N}{\mathbb{N}}
\newcommand{\Z}{\mathbb{Z}}
\newcommand{\Q}{\mathbb{Q}}
\newcommand{\R}{\mathbb{R}}
\newcommand{\Cm}{\mathbb{C}}
\newcommand{\F}{\mathbb{F}}
\newcommand{\id}{\mathrm{id}}
\newcommand{\imp}[2]{
	(#1\,\,$\ra$\,\,#2)\,\,
}
\newcommand{\Root}[2]{
	\left\{\!\sqrt[#1]{#2}\right\}
}
\newcommand{\RR}{\R}
\newcommand{\NN}{\N}
\renewcommand{\subseteq}{\subset}
\newcommand{\sub}{\subset}
\newcommand{\sconstr}{\;\vert\;}
\newcommand{\thus}{\implies}

\newcommand{\defeq}{\vcentcolon= }
\newcommand{\defev}{\stackrel{\Delta}{\Longleftrightarrow}}
\newcommand{\deriv}[3][1]{%
	\ifthenelse{#1>1}{%
		\frac{\dlta^{#1} {#2}}{\dlta {#3}^{#1}}
	}{%
		\frac{\dlta {#2}}{\dlta {#3}}
	}%
}

\renewcommand\labelitemi{$\triangleright$}

\let\bs\backslash
\let\lra\Leftrightarrow
\let\ra\Rightarrow
\let\la\Leftarrow
\let\emb\hookrightarrow

%%% Перенос знаков в формулах (по Львовскому)
\newcommand{\hm}[1]{#1\nobreak\discretionary{}{\hbox{$\mathsurround=0pt #1$}}{}}

%%% Работа с картинками
\usepackage{graphicx}    % Для вставки рисунков
\setlength\fboxsep{3pt}  % Отступ рамки \fbox{} от рисунка
\setlength\fboxrule{1pt} % Толщина линий рамки \fbox{}
\usepackage{wrapfig}     % Обтекание рисунков текстом

% \usepackage[inkscapeformat=png]{svg} %% svg

%%% Работа с таблицами
\usepackage{array,tabularx,tabulary,booktabs} % Дополнительная работа с таблицами
\usepackage{longtable}                        % Длинные таблицы
\usepackage{multirow}                         % Слияние строк в таблице

%%% Теоремы
\theoremstyle{plain}
\newtheorem*{theorem}{Теорема}
\newtheorem*{lemma}{Лемма}
\newtheorem*{proposition}{Утверждение}
\newtheorem*{exercise}{Упражнение}
\newtheorem*{problem}{Задача}

\theoremstyle{definition}
\newtheorem*{definition}{Определение}
\newtheorem*{corollary}{Следствие}
\newtheorem*{note}{Замечание}
\newtheorem*{reminder}{Напоминание}
\newtheorem*{example}{Пример}

\theoremstyle{remark}
\newtheorem*{solution}{Решение}

%%% Оформление страницы
\usepackage{extsizes}     % Возможность сделать 14-й шрифт
\usepackage{geometry}     % Простой способ задавать поля
\usepackage{setspace}     % Интерлиньяж
\usepackage{enumitem}     % Настройка окружений itemize и enumerate
\setlist{leftmargin=10pt} % Отступы в itemize и enumerate

\geometry{top=15mm}    % Поля сверху страницы
\geometry{bottom=5mm} % Поля снизу страницы
\geometry{left=10mm}   % Поля слева страницы
\geometry{right=10mm}  % Поля справа страницы

\setlength\parindent{15pt}        % Устанавливает длину красной строки 15pt
\linespread{1}                  % Коэффициент межстрочного интервала
%\setlength{\parskip}{0.5em}      % Вертикальный интервал между абзацами
\setcounter{secnumdepth}{0}      % Отключение нумерации разделов
%\setcounter{section}{-1}         % Нумерация секций с нуля
\usepackage{multicol}			  % Для текста в нескольких колонках
\usepackage{soulutf8}             % Модификаторы начертания
\mathtoolsset{showonlyrefs=true} % показывать номера формул только у тех, у которых есть ссылки по eqref
%%% Содержаниие
% \usepackage{tocloft}
% \tocloftpagestyle{main}
%\setlength{\cftsecnumwidth}{2.3em}
%\renewcommand{\cftsecdotsep}{1}
%\renewcommand{\cftsecpresnum}{\hfill}
%\renewcommand{\cftsecaftersnum}{\quad}

%%% Нумерация уравнений
\makeatletter
\def\eqref{\@ifstar\@eqref\@@eqref}
\def\@eqref#1{\textup{\tagform@{\ref*{#1}}}}
\def\@@eqref#1{\textup{\tagform@{\ref{#1}}}}
\makeatother                      % \eqref* без гиперссылки
\numberwithin{equation}{section}  % Нумерация вида (номер_секции).(номер_уравнения)
\mathtoolsset{showonlyrefs= true} % Номера только у формул с \eqref{} в тексте.

%%% Гиперссылки
\usepackage{hyperref}
\usepackage[usenames,dvipsnames,svgnames,table,rgb]{xcolor}
\hypersetup{
	unicode=true,            % русские буквы в раздела PDF
	colorlinks=true,       	 % Цветные ссылки вместо ссылок в рамках
	linkcolor=black!15!blue, % Внутренние ссылки
	citecolor=green,         % Ссылки на библиографию
	filecolor=magenta,       % Ссылки на файлы
	urlcolor=NavyBlue,       % Ссылки на URL
}

%%% Графика
\usepackage{tikz}        % Графический пакет tikz
\usepackage{tikz-cd}     % Коммутативные диаграммы
\usepackage{tkz-euclide} % Геометрия
\usepackage{stackengine} % Многострочные тексты в картинках
\usetikzlibrary{angles, babel, quotes}
\title{\texttt{Исследование взаимной диффузии газов \\ 2.2.1}}
\author{}
\date{}

\begin{document}
  \maketitle

\textbf{Цель работы:} 
\begin{itemize}
  \item  Регистрация зависимости концентрации гелия в воз-
  духе от времени с помощью датчиков теплопроводности при разных
  начальных давлениях смеси газов; 
  \item Определение коэффициента диф-
  фузии по результатам измерений.
\end{itemize}
\textbf{В работе используются:}  измерительная установка; форвакуумный
насос; баллон с газом (гелий); манометр; источник питания; магазин
сопротивлений; гальванометр; секундомер.

\section*{\texttt{Теория}}
Плотность потока вещества любого компонента в результате взаимной диффузии
\begin{equation}
  j_a = -D_{ab}\frac{\partial \eta_a}{\partial x}, 
  j_b = -D_{ba}\frac{\partial \eta_b}{\partial x}
\end{equation}
где $D_{ab} = D_{ba}$ — коэффициент взаимной диффузии компонентов,
$j_a, j_b$ — плотности потока частиц соответствующего сорта
\section*{Экспериментальная установка}
Схема экспериментальной установки представлена на рис. Поток воздуха поступает через газовый счетчки в металлические трубки, трубки имеют заглушки и отверстия для подключения микроманометра.  \ref{установка}. 

\section*{Ход работы}

Данные изменрений давления от расхода $\Delta P (Q)$ приведены ниже. 

% \begin{center}
%   Результаты измерений разности давлений от расхода $\Delta P (Q)$.  
% \end{center}
\begin{table}[!htb]
  \begin{minipage}{.5\linewidth}
    \caption{Для трубки $d_1$}
    \centering
      \begin{tabular}{|c|c|}
      \hline
      $Q$, мл/c & $\Delta P$, Па \\ \hline
      06,4 &   9,81 \\ \hline
      17,8 &  29,42 \\ \hline
      31,0 &  49,04 \\ \hline
      44,4 &  68,65 \\ \hline
      57,2 &  88,26 \\ \hline
      69,7 & 107,88 \\ \hline
      84,0 & 127,49 \\ \hline
      85,0 & 133,38 \\ \hline
      92,3 & 147,11 \\ \hline
      98,0 & 166,72 \\ \hline
      102,7 & 186,33 \\ \hline
      104,2 & 205,95 \\ \hline
      106,4 & 225,56 \\ \hline
      109,9 & 245,18 \\ \hline
      \end{tabular}
  \end{minipage}
  \begin{minipage}{.5\linewidth}
    \centering
      \begin{tabular}{|c|c|}
      \hline
      $Q$, мл/c &  $\Delta P$, Па \\ \hline
        27,0 &          9,81 \\ \hline
        86,0 &         29,42 \\ \hline
        52,0 &         19,61 \\ \hline
        116,8 &         39,23 \\ \hline
        138,5 &         49,04 \\ \hline
        149,7 &         58,84 \\ \hline
        158,7 &         68,65 \\ \hline
        165,0 &         78,46 \\ \hline
        171,0 &         88,26 \\ \hline
        203,0 &        127,49 \\ \hline
        233,0 &        166,72 \\ \hline
        263,0 &        205,95 \\ \hline
        285,0 &        239,29 \\ \hline
      \end{tabular}
  \end{minipage} 
\end{table}

Ниже на графике крестами выделены точки, начиная с которых начинается турбулетное течение. С помощью МНК построим линейную зависимость для точек, которые предположительно принадлежат ламинарному течению (график и его анализ представлены ниже).

Из формулы Пуазейля имеем: 
% \[\eta = \frac{k \pi R^4}{8 l}\]
% \[Q = \frac{\pi R^4 \cdot \Delta P}{8 \eta l}\]
\[\Delta P = Q \cdot \frac{8 \eta l}{\pi R ^4} = Q \cdot k\]

Тогда для погрешности систематической получаем (с учётом, что погрешность $\Delta P$ есть погрешность измерения высоты столба воды, то есть равна $\sigma_{\Delta P} = 0,2 \cdot 9,807 \cdot  \sigma_h = 1,9$ Па): 
\[\sigma_k^{\text{сист}} = k \cdot \sqrt{\left(\frac{\sigma_{\Delta P}}{{\Delta P}_{\max}}\right)^2 + \left(\frac{\sigma_Q}{Q_{\max}} \right)^2}\]


Из МНК имеем, что:

\begin{gather}
  k = \frac{\langle Q \cdot \Delta P \rangle - \langle Q \rangle \cdot \langle \Delta P \rangle}{\langle {Q}^2 \rangle - {\langle Q \rangle}^2} \\
  \sigma_k^{\text{случ}} = \sqrt{\frac{1}{N} \cdot \left(\frac{\langle {\Delta P}^2 \rangle}{\langle Q^2 \rangle} - k^2 \right)} \\
  \sigma_k = \sqrt{(\sigma_k^{\text{случ}})^2 + (\sigma_k^{\text{сист}})^2} \\
  \sigma_\eta = \eta \sqrt{{\left(\frac{\sigma_k}{k}\right)}^2 + 4^2 \cdot {\left(\frac{\sigma_R}{R}\right)}^2 + {\left(\frac{\sigma_l}{l}\right)}^2}
\end{gather}

% \[k = \frac{\langle Q \cdot \Delta P \rangle - \langle Q \rangle \cdot \langle \Delta P \rangle}{\langle {Q}^2 \rangle - {\langle Q \rangle}^2}\]
\begin{table}
  \caption{Результаты из МНК и графика}
  \centering
    \begin{tabular}{|c|c|c|}
      \hline
      & $d_1$ & $d_2$ \\ \hline
      $k \cdot 10^6 $, $\text{Па $\cdot$ с / м $^3$ }$ & 1,53 & 0,37 \\ \hline
      $\sigma_k^{\text{случ}} \cdot 10^6$, $\text{Па $\cdot$ с / м $^3$ }$ & 0,02 & 0,03 \\ \hline
      $\sigma_k^{\text{сист}} \cdot 10^6$, $\text{Па $\cdot$ с / м $^3$ }$ & 0,02 & 0,01 \\ \hline
      $\sigma_k \cdot 10^6$, $\text{Па $\cdot$ с / м $^3$ }$ & 0,03 & 0,03 \\ \hline
      $\eta \cdot 10^-6$, Па $\cdot$ с & 6,96 & 8,05 \\ \hline
      $\sigma_\eta \cdot 10^-6 $, Па $\cdot$ с & 0,38 & 0,71 \\ \hline 
    \end{tabular}
\end{table}

То есть окончательно имеем, что: 
\begin{align*}
  \eta_1 = (6,96 \pm 0,38) \cdot 10^{-6} \text{Па $\cdot $ с} \\
  \eta_2 = (8,05 \pm 0,71) \cdot 10^{-6} \text{Па $\cdot $ с}
\end{align*}

Далее найдём значение $Re = \frac{\rho Q}{\pi R \eta}$, $\sigma_{Re} = Re \cdot \sqrt{{\left( \frac{\sigma_Q}{Q} \right)} ^ 2 + {\left( \frac{\sigma_R}{R} \right)} ^ 2 + {\left( \frac{\sigma_\eta}{\eta} \right)} ^ 2}$ для каждой трубки.

\begin{itemize}
  \item $d_1$: $Q_\text{крит} = 0,085$ л/ с. То есть $Re = 2304 \pm 140$. 
  \item $d_2$: $Q_\text{крит} = 0,149$ л/ с. То есть $Re = 2370 \pm 208$.
\end{itemize}


Далее определим длину участка трубы, на котором происходит уставновление потока. Построим график $P(x)$ для каждой трубы. Будем проводить прямую через все точки кроме первой, считая, что на них уже должно было почти установиться течение.

\newpage
\begin{itemize}
  \item $d_1$. Ожидаемая длина: $l_{\text{уст}} = 0,2 \cdot R_1 \cdot Re = 0,91$ м.
  \item $d_2$. Ожидаемая длина: $l_{\text{уст}} = 0,2 \cdot R_2 \cdot Re = 1,39$ м.
\end{itemize}

\section*{Вывод}

Экспериментально исследовались свойства течения газов по тонким трубкам при различ-
ных числах Рейнольдса; выявить область применимости закона Пуазейля и с его помощью
определить коэффициент вязкости воздуха. Получили вязкость воздуха:
\begin{align*}
  \eta_1 = (6,96 \pm 0,38) \cdot 10^{-6} \text{Па $\cdot $ с} \\
  \eta_2 = (8,05 \pm 0,71) \cdot 10^{-6} \text{Па $\cdot $ с}
\end{align*}
Учтя, что $\eta_\text{табл} = (1,3) \cdot 10^{-6} $Па $\cdot $ с. Это значение отличается от табличного, возможно это связанно с тем, что был не правильно выбран "нулевой" уровень для измерения давления.

Число Рейнольдса получилось $Re_1 = 2304 \pm 140$ и $Re_2 = 2370 \pm 208$, что тоже отличается от $Re \sim 1000$, что скорее всего также связанно с неправильным выбором "нулевого" уровня.


%   \caption{Зависимость $\Delta P(Q)$.}
%   \caption{Зависимость $P(x)$.}



\end{document}

