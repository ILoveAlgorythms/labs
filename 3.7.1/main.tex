% !TeX root = main.tex
\documentclass{article}
%\usepackage{blindtext}
%\usepackage[a4paper, total={6in, 9.4in}]{geometry}

% \input{../preamble.tex}
\usepackage{wrapfig}
\usepackage{graphicx}
\usepackage{mathtext}
\usepackage{amsmath}
\usepackage{siunitx} % Required for alignment
\usepackage{subfigure}
\usepackage{multirow}
\usepackage{rotating}
\usepackage{afterpage}
\usepackage[T1,T2A]{fontenc}
\usepackage[russian]{babel}
\usepackage{caption}
\usepackage[arrowdel]{physics}
\usepackage{booktabs}
\usepackage{float}


% \graphicspath{{pictures/}}

\begin{document}
\title{\begin{center}Лабораторная работа №3.7.1\end{center}
	Скин-эффект в полом цилиндре}
\author{Александр Ляпин, Кирилл Нелюбин, Б05-207}
\date{\today}

	\maketitle
	\section*{Цель работы:} Исследование проникновения переменного магнитного поля в медный полый цилиндр
	
	\section*{Экспериментальная установка}
\begin{figure}[H]
	\center
  \includegraphics*[width=0.7\textwidth]{24_12_48.png}
  \caption*{Экспериментальная установка для изучения скин-эффекта}
\end{figure}
\begin{figure}[H]
	\center
  \includegraphics*[width=0.7\textwidth]{24_12_59.png}
  \caption*{Схема подключения RLC-метра}
\end{figure}

	\section*{Ход работы}
	
	Параметры нашей установки $2a = 45$ мм, $h=1.5$ мм. Проводимость порядка
	$\sigma \sim 5\cdot 10^7$ См/м. Получаем оценку для частоты, при которой
	глубина проникновения равна толщине стенок цилиндра $\nu_h = 2254$ Гц.
	
	\subsection*{Измерения амплитуд в области низких частот}
	В области низких частот толщина скин-слоя превосходит толщину образца $ \delta \gg h$  и из (\ref{eq:svyaz_poley}) получаем
	\begin{equation*}
		\left(\frac{|H_1|}{|H_0|}\right)^2 = (\xi_0\xi)^2 \approx \frac{1}{1+\left(\frac{ah}{\delta^2}\right)^2} = \frac{1}{1 + \left(\pi ah\nu\mu_0\sigma\right)^2}
	\end{equation*}
	Тогда: 
	\begin{equation*}
		\frac{1}{\xi^2}=\xi_0^2B^2\nu^2 + \xi_0^2 \text{, где } B=\pi a h \sigma \mu_0
		\label{eq:liniya_dlya_c}
	\end{equation*}
	\begin{figure}[h!]
		\centering
		\includegraphics[width=0.9\textwidth, height = 0.45\textheight]{15_13_25.png}
		\caption{График зависимости $1/\xi^2(\nu^2)$}\label{fig:xi_nu_low_freq_linearized}
	\end{figure}
	Получаем следующие значения: $\xi_0^2B^2 = 0.138, \ \xi_0^2 = 4212.65$, тогда:
	\[\xi_0 = 64.90 \pm 0.04 \ \frac{\text{Гц}}{\text{Ом}}, \ \sigma = (4.294 \pm 0.005) \cdot 10^7 \ \frac{\text{См}}{\text{м}}  \]
	
	\subsection*{Измерение проводимости через разность фаз при низких частотах}
	Построим график $\tg{\psi} (\nu)$ по тем точкам точкам, для которых он хорошо аппроксимируется прямой (при $\nu \approx 0.5 \nu_h \ \tg \psi \rightarrow +\infty$) 
	Согласно формуле (\ref{eq:faza_low_freq}), при $\delta \gg h$
	\begin{equation*}
		\tg \psi = \frac{ahw \sigma \mu_0}{2} = \pi ah\mu_0\sigma \nu \ \ (\mu = 1)
	\end{equation*}
	Коэффициент наклона прямой: \[\pi ah \mu_0\sigma = k = (5.2 \pm 1) \cdot 10^{-3} \ \text{с}\]
	\[\sigma = \frac{k}{\pi ah \mu_0} = (3.93 \pm 0.73) \cdot 10^7 \ \frac{\text{См}}{\text{м}}\]
	\begin{figure}[H]
		\centering
		\includegraphics[width=0.7\textwidth]{15_13_28.png}
		\caption{График зависимости $\tg \psi (\nu)$}\label{fig:tg_psi_nu_line}
	\end{figure}

	\subsection*{Измерение проводимости через разность фаз в высокочастотном диапазоне}
	Согласно формуле (\ref{eq:faza_high_freq}), при $\delta \ll h$
	\begin{equation*}
		\psi - \pi/4 = k\cdot \sqrt{\nu}; \ k = h\sqrt{\pi\mu_0\sigma}
	\end{equation*}
	
	Получено значение $k = 0.0184 \pm 0.0014$, отсюда получаем значение проводимости:
	
	\begin{equation}
		\sigma = (3.80 \pm 0.58) \cdot 10^7 \ \frac{\text{См}}{\text{м}}
	\end{equation}
	
	\begin{figure}[h]
		\centering
		\includegraphics[width=\textwidth]{15_13_30.png}
		\caption{График зависимости $(\psi - \pi/4)(\sqrt{\nu})$}
		\newpage
	\end{figure}
	
	\subsection*{Измерение проводимости через изменение индуктивности}
	Измерить проводимость можно также через изменение индуктивности катушки внутри цилиндра. Данные, измеренные с помощью $RCL$-метра:
	
	\begin{table}[h!]
		\centering
		\begin{tabular}{|c|c|c|c|c|c|c|c|c|c|c|c|c|c|c|}
			\hline
			$\nu$, кГц & 0.04 & 0.15 & 0.25 & 0.3 & 0.4 & 0.5 & 0.6 & 0.8 & 1.5 & 2.5 & 4 & 10 & 15 & 20 \\
			\hline
			$L$, мГн & 10 & 7.35 & 5.4 & 4.8 & 4.0  & 3.65 & 3.45 & 3.26 & 2.9 & 2.9 & 2.9 & 3 & 3.17 & 3.6 \\
			\hline
		\end{tabular}
		\caption{Значения индуктивности катушки при различных частотах}
	\end{table}
	
	Примерно так выглядит график $L(\nu)$:
	
	\begin{figure}[H]
		\centering
		\includegraphics[width = 0.9\textwidth]{15_13_31.png}
		\caption{График зависимости $L(\nu)$}
	\end{figure}
	
	Полученные максимальные и минимальные значения: $L_{min} = 2.9$ мГн, $L_{max} = 10$ мГн.
	\begin{equation*}
		\frac{L_{\max} - L}{L - L_{\min}} = \pi ^2 a^2 h^2 {\mu_0}^2 \sigma^2 \nu^2
	\end{equation*}
	
	То есть коэффициент наклона графика
	\[k = (\pi ah\mu_0 \sigma)^2 \ \rightarrow \sigma = \frac{\sqrt{k}}{\pi ah \mu_0}\]
	
	Подставляя полученные значения, получаем:
	
	\begin{equation}
		\sigma = (4.11 \pm 0.07) \cdot 10^7  \ \frac{\text{См}}{\text{м}}
	\end{equation}
	
	\begin{figure}[h!]
		\centering
		\includegraphics[width=\textwidth]{15_13_32.png}
		\caption{График зависимости $\frac{L_{\max} - L}{L - L_{\min}} (\nu^2)$}
	\end{figure}
	
	
	\subsection*{Отношение магнитных полей}
	Отношение $\abs{H_1}/\abs{H_0}$ можем посчитать двумя способами. Первый способ - через
	формулу (\ref{eq:otnoshenie_amplitud}),использовав посчитанное значение $\xi_0$ в анализе амплитуд в области низких частот.
	Второй способ - через теоретическую формулу (\ref{eq:svyaz_poley}), использовав первое полученное значение $\sigma$. Посмотрим на их различие с помощью графиков зависимости
	$\abs{H_1}/\abs{H_0} (\nu)$
	
	\begin{figure}[h]
		\centering
		\includegraphics[width=\textwidth]{15_13_33.png}
		\caption{График зависимость $\frac{|H_1|}{|H_0|}(\nu)$}
	\end{figure}
	
	\section*{Выводы}
	В данной лабораторной работе мы измеряли удельную проводимость меди 4-мя различными способами с помощью явления скин-эффекта. Запишем результаты в общую таблицу:
	
	\begin{table}[!h]
		\begin{center}
			\begin{tabular}{|l|c|c|c|}
				\hline
				Метод измерения & $\sigma, 10^{7} \ \frac{\text{См}}{\text{м}}$ & $\Delta\sigma, 10^{7} \ \frac{\text{См}}{\text{м}}$ & $\varepsilon_{\sigma}$\\
				\hline
				Отношение амплитуд & 4.294 & 0.005 & 0.1\%\\ \hline
				Разности фаз (низкие частоты) & 3.93 & 0.73 & 18.6\%\\ \hline
				Разности фаз (высокие частоты) & 3.80 & 0.58 & 15.2\%\\ \hline
				Индуктивность & 4.11 & 0.07 & 1.8\%\\ \hline
				
			\end{tabular}
		\end{center}
		\caption{Сравнение результатов различных методов}\label{}
	\end{table}
	
	В работе использовалась медь марки $M3$, для которой $\sigma_{\text{табл}} = 5.62\cdot10^{7} \ \frac{\text{См}}{\text{м}}$.
	Полученные нами значения совпадают по порядку, но, все же, немного нижу табличного значения. Несовпадение может быть вызвано многими факторами, например наводкой поля в соединительных проводах и пренебрежением размерами медного цилиндра и соленоида. 
	
	Методы измерения через разность фаз дали высокие погрешности, потому что измерения делались на глаз на осциллографе, и гарантировать их точность можно только с введенной погрешностью. Кроме того, при измерении на высоких частотах зависимость не является везде линейной, это тоже привносит свою неточность.
	
	Что касается зависимости $\frac{|H_1|}{|H_0|}(\nu)$, то экспериментальные данные очень хорошо согласуются с теоретической зависимостью.
\end{document}