% !TeX root = main.tex
\documentclass[12pt,a4paper]{article}
\input{../preamble.tex}
\newcommand{\mk}{\mathfrak}
%%% значок в itemize
\renewcommand{\labelitemi}{$\multimap  $}
\title{\texttt{Скин-эффект \\ 3.7.1}}
\author{Кирилл Нелюбин, Б05-207}
\date{\today}

\begin{document}
\maketitle

\section{Цель работы:}
исследовать явление проникновение переменного магнитного поля в медный полый цилиндр
\section{В работе используются:}
генератор сигналов АКИП–3420, соленоид, намотанный на полый
цилиндрический каркас, медный экран в виде полого цилиндра, измерительная катушка, ам
перметр, вольтметр, двухканальный осциллограф GOS–620, RLC-метр.

\section*{Экспериментальная установка}
\begin{figure}[H]
  \includegraphics*[width=0.7\textwidth]{24_12_48.png}
  \caption*{Экспериментальная установка для изучения скин-эффекта}
\end{figure}
\begin{figure}[H]
  \includegraphics*[width=0.7\textwidth]{24_12_59.png}
  \caption*{Схема подключения RLC-метра}
\end{figure}
% \subsection*{\texttt{Важные константы:}}
% 24_14_14.png
\section{Ход работы}
\subsection{Определение магнитного момента магнитных шариков}
  \begin{figure}[H]
    \begin{tabular}{|r|c|r|}
       $n$ & $m_\text{гр}, \text{гр} \pm 0.003$ &$r_\text{гр},$ см$\pm 0.01$\\
      \midrule
       4 & 0.391 & 0.59\\
       6 & 0.267 & 1.19\\
       8 & 0.182 & 1.78\\
      10 & 0.119 & 2.38\\
    \end{tabular}
    \caption{Зависимость $m_\text{гр}(n), r_\text{гр}(n)$}      
  \end{figure}
\section*{Вывод}

\end{document}
