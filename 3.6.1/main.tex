% !TeX root = main.tex
\documentclass[14pt,a4paper]{article}
\input{../preamble.tex}
%%% значок в itemize
\renewcommand{\labelitemi}{$\multimap  $}
\title{\texttt{Спектральный аналих электрических сигналов\\ 3.6.1}}
\author{\texttt{Кирилл Нелюбин, Б05-207}}
\date{\texttt{\today}}
\begin{document}
\maketitle

% \textbf{Цель работы:}
% \begin{itemize}
%   \item  Изучение вольт-амперной характеристики тлеющего раз­
%   ряда
%   \item Изучение свойств плазмы методом зондовых характеристик.
% \end{itemize}

\textbf{В работе используются:}  
понижающий трансформатор, реостат, резистор, интегрирующая цепочка, амперметр и вольтметр (мультиметры), электронный оcциллогра, делитель напряжения, переключатель, тороидальные образцы с дву-
мя обмотками.


\section*{Экспериментальная установка}
% \subsection*{\texttt{Важные константы:}}

% \begin{figure}[H]
%   \includegraphics*[width=0.8\textwidth]{2023-09-30-11-06-33.png}
%   \caption{Схема установки для исследования намагничевания образцов}
%   \label{fig:ust}
% \end{figure}
% \subsection*{Параметры установки}
% \begin{itemize}
%   \item $R_\text{и} = 20$ кОм
%   \item $C_\text{и} = 20$ мкФ
%   \item $R_0 = 0.22$ Ом
% \end{itemize}

% \noindent Схема экспериментальной установки представлена на рис~\ref{fig:ust}.
\newpage
\section*{Ход работы}
\subsection*{А}
\begin{enumerate}
  % \setcounter{enumi}{-1}
  \item Пики едут вправо при увеличении частоты
  \item Пики растут вверх
\end{enumerate}

\subsection*{Проверка калибровки осциллографа}
\begin{enumerate}
  \item \textit{Проведём калибровку горизонтальной оси ЭО.} Закоротим обмотку $N_0$, с помощью реостата  $R_1$ подберём такой ток, чтобы горизонтальная прямая занимала большую часть экрана ЭО. По формуле:
  $$K_X=2R_0\sqrt{2}I_\text{ЭФ} / 2x$$
  \footnotesize формула берётся из соображений перевода показаний амперметра в показания ЭО \normalsize\\
  (2x -- длина горизонтальной прямой) расчитаем $K_X$. Повторим измерения для всех $K_X$, которые мы использовали. Сведём результаты в табличку:
  
  \item \textit{Проведём калибровку вертикальной оси ЭО.} Соединим
  вход Y ЭО с клеммами делителя 1/100 и общий (земля), переведём $K_0$ в положение Д. Не меняя рабочего коэициента $K_Y$, подберём с
  помощью реостата $R_1$ напряжение, при котором вертикальная прямая занимает почти весь экран. Измерим длину прямой 2y в см (двойную амплитуду
  сигнала).
  Подключим мультитметр V его к тем же
  точкам делителя и определим эективное значение напряжения.
  Запишите напряжение $U_\text{ЭФ}$, величину сигнала на экране 2y в см и коэфициент усиления осциллограа $K_Y$. По формуле:
  $$K_Y=2\sqrt{2}U_\text{ЭФ} / 2y$$
  \footnotesize формула берётся из соображений перевода показаний амперметра в показания ЭО \normalsize\\
  (2x -- длина горизонтальной прямой) расчитаем $K_X$.
  Повторим измерения для всех $K_Y$, которые мы использовали. Сведём результаты в табличку (см приложение, рис \ref{x}, \ref{y})
\end{enumerate}


\subsection*{Обработка результатов}
\begin{enumerate}
  \item Расчитаем коэфициенты преобразования отклонений по осям ЭО в напряженность $H$ и индукцию $B$ по формулам:
  $$H = \frac{I}{N_0}{2\pi R}, \ \ \ \ \ \ B = \frac{R_\text{и}C_\text{и}}{SN\text{и}}U_\text{вых}$$
  \item Рассчитаем амплитуду
  $H_{max}$, соответствующую состоянию насыщения (предельное поле). Вычислим индукцию насыщения $B_s$.
  \item Рассчитаем коэрцитивное поле $H_c$ и остаочную индукцию $B_r$ для каждого образца. По начальным кривым намагничивания оцените начальные и максимальные
значения диеренциальной магнитной проницаемости $\mu_\text{диф}$ = $dB/dH$.
  Сведём в таблицу:
  \begin{figure}[H]
    \begin{tabular}{l|ccc}
      {} &        Fe-Ni &            Fe &         Fe-Si \\
      \midrule
      N0  &    15 &     45 &     20 \\
      Nu  &   300 &    400 &    200 \\
      $S, \text{см}^2$   &     0.66 &      3.00 &      2.00 \\
      $2\pi R,$ см &     14.10 &      25.00 &      11.00 \\
      I, А   &     0.229 &      0.937 &      1.298 \\
      $K_y,$ мВ  &     50 &      50 &      20 \\
      H, A/м   &  2 436 &  1 6878 &  23 596 \\
      B, Тл   &     1.010 &      0.167 &      0.200 \\
      \end{tabular}
  \end{figure}
\end{enumerate}


\subsection*{Определение характерного времени интегрерующей ячейки}\begin{enumerate}
  \item[0] Формула характерного времени:
  $$\tau = RC = \frac{U_\text{ВХ}}{\omega U_\text{ВЫХ}}$$
  \item Подадим на вход ячейки
  напряжение с обмотки 6,3 B трансформатора ($K_0$ в положении П).
  \item Подключим канал Y ЭО ко входу интегрирующей ячейки, отключим канал X. Установите чувствительность $K_Y \sim$ n V /дел. Подберём с помощью реостата $R_1$ такой ток, при котором вертикальная прямая занимает
  большую часть экрана, и определим входное напряжение на RC-цепочке:
  $$U_\text{ВХ} = 2y \cdot K_Y$$
  Не меняя тока, переключим Y-вход ЭО к интегрирующей ёмкости и
  аналогичным образом определим выходное напряжение $U_\text{ВЫХ}$.
  \item  Рассчитаем на месте постоянную времени по формуле из нулевого пункта:\\
  $U_\text{ВХ} = 2 * 4 * 1$В, \ \  $U_\text{ВЫХ} = 0.055$В => \underline{$\tau = 0.46$с}.\\
  Через $RC:$ $\tau = 0.40$с => отклонение -- 13\%
\end{enumerate}

\section*{Вывод}

\newpage
\section*{Приложение}
\end{document}
