% !TeX root = main.tex
\documentclass[12pt,a4paper]{article}
\usepackage{caption}
\input{../preamble.tex}
\newcommand{\mk}{\mathfrak}
\newcommand{\s}[1]{\frac{\sigma_{#1}}{#1}}
\newcommand{\sq}[2]{\frac{\sigma_{#1}^{#2}}{#1^{#2}}}
%%% значок в itemize
\renewcommand{\labelitemi}{$\multimap  $}

\title{
  \newlength{\imageheight}
  \settoheight{\imageheight}{A}
  \texttt{Опыт Миликена\includegraphics*[height=0.7\imageheight]{10_12_22.png}\\ 3.3.3}
  }
\author{Александр Ляпин, Кирилл Нелюбин, Б05-207}
\date{\today}

\begin{document}
\maketitle
\section{Цель работы:}
Обнаружение дискретности электрического заряда
\section{В работе используются:}  
Плоский конденсатор в защитном кожухе, осветитель, измерительный микроскоп, выпрямитель, электростатический вольтметр, переключатель напряжения, пульверизатор с маслом, секундомер.

\section*{\texttt{Важные формулы}}\noindent
Формула заряда капли:
\begin{align}
  q=9\pi\sqrt{\frac{2 \eta^3 h^3}{\rho g}}
    \frac{t_0+t}{Vt_0^{3/2}t}
    \label{q}
\end{align}
Погрешность:
\begin{align}
	\s{q} =\sqrt{
    \sq V 2 + \frac 1 4\sq t 2 + \sq{t_0} 2
  }
    \label{sigma_q}
\end{align}



\section*{Экспериментальная установка}
\begin{figure}[H]
  \includegraphics*[width=0.8\textwidth]{10_10_26.png}
  \label{ust}
  \caption{Схема экспериментальной установки для измерения заряда электрона}
\end{figure}


\section{Ход работы}
\begin{enumerate}
  \setcounter{enumi}{-1}
  \item Вычислим из формулы для заряда капли (в начаде), $V\thickapprox 1$кВ.
  \item Включим осветитель, настроим окуляр на резкое изборажение.
  \item Аккуратно нажмём на пульверизатор, выберем 1 каплю.
  \item С помощью секундомера засечём время её падения на 1 мм, включим напряжение, засечём время её падения на 1 мм при включенном напряжении, запишем напряжение. Для каждой капли повторим 3-5 раз для уменьшения случайной погрешности измерения времени.
  \item Повторим предыдущий пункт для 14 капель, результаты сведём в таблицу (в приложении)
  \item Вычислим заряды капель:
  \begin{figure}[h!]
    \includegraphics*[width=0.95\textwidth]{17_14_01.png}
    \caption*{заряды капель, $\cdot 10^{-19}$Кл}
  \end{figure}
  \item Разобьём на группы, возьмём в каждой медиану, поделим каждую медиану на первую:
  \begin{figure}[h!]
    \begin{tabular}{r}
      1.00 \\
      3.80 \\
      6.76 \\
      7.41 \\
      8.19 \\
      \end{tabular}
      \caption*{\texttt{Заряды капель к наименьшему заряду}}
  \end{figure}
  \item Оценим максимальный путь релаксации по формуле:
  $$s \sim \frac{1}{g} \left( \frac{h}{t_0}\right)^2$$
  \underline{s = $8.2\cdot 10^{-2}$нм}.
\end{enumerate}


\section*{Вывод}
Предположение о дискретности заряда не подтверждается. Помешало наблюдение броуновское движение капель (можно убдедиться, исследовав разборс $t$ для отдельной капли)\\
Путь релаксации: s = $8.2\cdot 10^-2$нМ.

\newpage
\section*{\texttt{Приложение}}
\noindent
\begin{figure}[h!]
  \begin{tabular}{rrrr}
\toprule
 n &    t0 &     t &    kV \\
\midrule
 1 & 15.65 & 13.25 &   480 \\
 1 & 16.33 & 14.13 &   480 \\
 1 & 14.16 & 14.13 &   460 \\
 1 & 17.38 & 15.25 &   460 \\
 1 & 15.85 & 13.51 &   490 \\
 3 & 23.74 &  9.11 &   470 \\
 3 & 22.00 &  9.20 &   460 \\
 3 & 21.76 &  7.83 &   470 \\
 3 & 21.00 &  8.05 &   470 \\
 3 & 21.08 &  8.74 &   480 \\
 4 & 21.88 &  8.38 &   470 \\
 4 & 21.18 &  7.55 &   470 \\
 4 & 24.73 &  7.23 &   470 \\
 4 & 25.38 &  8.15 &   470 \\
 4 & 23.76 &  8.26 &   470 \\
 5 & 20.25 &  8.25 &  2000 \\
 5 & 21.00 &  7.71 &  2000 \\
 5 & 19.00 &  8.36 &  2000 \\
 5 & 18.45 &  8.48 &  2000 \\
 5 & 18.73 &  8.06 &  2000 \\
 6 & 16.13 & 15.25 & -2200 \\
 6 & 18.03 & 15.48 & -2200 \\
 6 & 16.21 & 16.81 & -2200 \\
 6 & 16.30 & 16.53 & -2200 \\
 6 & 15.38 & 15.56 & -2200 \\
 7 & 13.35 & 15.56 &  -290 \\
 7 & 12.55 & 15.06 &  -290 \\
 7 & 12.51 & 14.66 &  -290 \\
 7 & 12.60 & 16.09 &  -280 \\
 7 & 13.88 & 14.78 &  -280 \\
\bottomrule
\end{tabular}

  \begin{tabular}{rrrr}
\toprule
 n &    t0 &     t &   kV \\
\midrule
 8 & 25.68 &  9.73 & -200 \\
 8 & 23.81 &  9.23 & -200 \\
 8 & 22.65 & 10.30 & -200 \\
 8 & 23.21 &  9.78 & -200 \\
 8 & 22.18 &  5.91 & -300 \\
 9 & 16.46 & 14.83 &  220 \\
 9 & 15.08 & 13.78 &  220 \\
 9 & 17.80 & 14.83 &  220 \\
 9 & 15.73 & 15.30 &  220 \\
 9 & 16.48 & 13.90 &  220 \\
10 & 19.20 & 13.80 & -340 \\
10 & 21.13 & 13.60 & -340 \\
10 & 21.48 & 15.23 & -340 \\
10 & 20.11 & 15.46 & -340 \\
10 & 20.93 & 14.18 & -340 \\
11 & 19.13 & 18.51 &  340 \\
11 & 17.46 & 18.11 &  340 \\
11 & 17.26 & 18.35 &  340 \\
11 & 18.38 & 16.53 &  340 \\
11 & 16.56 & 21.65 &  340 \\
12 & 11.81 & 11.56 & -340 \\
12 & 12.15 & 12.83 & -340 \\
12 & 11.85 & 11.16 & -340 \\
12 & 11.80 & 11.75 & -340 \\
12 & 12.00 & 11.56 & -340 \\
13 & 35.35 & 13.28 &  250 \\
13 & 29.95 & 14.75 &  250 \\
13 & 29.91 & 12.46 &  250 \\
14 & 13.05 &  9.00 & -340 \\
14 & 14.18 & 20.53 & -220 \\
14 & 12.71 & 21.00 & -220 \\
\bottomrule
\end{tabular}

  % \label{hui}
  \caption*{Измерение капель}
\end{figure}

\end{document}