% !TeX root = main.tex
\documentclass[12pt,a4paper]{article}
\usepackage{caption}
% \documentclass[a4paper,14pt, draft]{article}

%%% отключение нумерации сраниц
\pagestyle{empty}
%%% значок в itemize
% \renewcommand{\labelitemi}{$\cdot$}

%%% Работа с русским языком
\usepackage{cmap}					% поиск в PDF
\usepackage{mathtext} 				% русские буквы в формулах
\usepackage[T1, T2A]{fontenc}			% кодировка %Т1 посоветовал чат гпт
\usepackage[utf8]{inputenc}			% кодировка исходного текста
\usepackage[english,russian]{babel}	% локализация и переносы
\usepackage{indentfirst}            % красная строка в первом абзаце
\frenchspacing                      % равные пробелы между словами и предложениями

%%% Дополнительная работа с математикой
\usepackage{amsmath,amsfonts,amssymb,amsthm,mathtools} % пакеты AMS
\usepackage{icomma}                                    % "Умная" запятая

%%% Свои символы и команды
\usepackage{centernot} % центрированное зачеркивание символа
\usepackage{stmaryrd}  % некоторые спецсимволы
\usepackage{dsfont}
\usepackage{amsthm}


\renewcommand{\epsilon}{\ensuremath{\varepsilon}}
\renewcommand{\phi}{\ensuremath{\varphi}}
\renewcommand{\kappa}{\ensuremath{\varkappa}}
\renewcommand{\le}{\ensuremath{\leqslant}}
\renewcommand{\leq}{\ensuremath{\leqslant}}
\renewcommand{\ge}{\ensuremath{\geqslant}}
\renewcommand{\geq}{\ensuremath{\geqslant}}
\renewcommand{\emptyset}{\ensuremath{\varnothing}}

\DeclareMathOperator{\sgn}{sgn}
\DeclareMathOperator{\ke}{Ker}
\DeclareMathOperator{\im}{Im}
\DeclareMathOperator{\re}{Re}

\newcommand{\N}{\mathbb{N}}
\newcommand{\Z}{\mathbb{Z}}
\newcommand{\Q}{\mathbb{Q}}
\newcommand{\R}{\mathbb{R}}
\newcommand{\Cm}{\mathbb{C}}
\newcommand{\F}{\mathbb{F}}
\newcommand{\id}{\mathrm{id}}
\newcommand{\imp}[2]{
	(#1\,\,$\ra$\,\,#2)\,\,
}
\newcommand{\Root}[2]{
	\left\{\!\sqrt[#1]{#2}\right\}
}
\newcommand{\RR}{\R}
\newcommand{\NN}{\N}
\renewcommand{\subseteq}{\subset}
\newcommand{\sub}{\subset}
\newcommand{\sconstr}{\;\vert\;}
\newcommand{\thus}{\implies}

\newcommand{\defeq}{\vcentcolon= }
\newcommand{\defev}{\stackrel{\Delta}{\Longleftrightarrow}}
\newcommand{\deriv}[3][1]{%
	\ifthenelse{#1>1}{%
		\frac{\dlta^{#1} {#2}}{\dlta {#3}^{#1}}
	}{%
		\frac{\dlta {#2}}{\dlta {#3}}
	}%
}

\renewcommand\labelitemi{$\triangleright$}

\let\bs\backslash
\let\lra\Leftrightarrow
\let\ra\Rightarrow
\let\la\Leftarrow
\let\emb\hookrightarrow

%%% Перенос знаков в формулах (по Львовскому)
\newcommand{\hm}[1]{#1\nobreak\discretionary{}{\hbox{$\mathsurround=0pt #1$}}{}}

%%% Работа с картинками
\usepackage{graphicx}    % Для вставки рисунков
\setlength\fboxsep{3pt}  % Отступ рамки \fbox{} от рисунка
\setlength\fboxrule{1pt} % Толщина линий рамки \fbox{}
\usepackage{wrapfig}     % Обтекание рисунков текстом

% \usepackage[inkscapeformat=png]{svg} %% svg

%%% Работа с таблицами
\usepackage{array,tabularx,tabulary,booktabs} % Дополнительная работа с таблицами
\usepackage{longtable}                        % Длинные таблицы
\usepackage{multirow}                         % Слияние строк в таблице

%%% Теоремы
\theoremstyle{plain}
\newtheorem*{theorem}{Теорема}
\newtheorem*{lemma}{Лемма}
\newtheorem*{proposition}{Утверждение}
\newtheorem*{exercise}{Упражнение}
\newtheorem*{problem}{Задача}

\theoremstyle{definition}
\newtheorem*{definition}{Определение}
\newtheorem*{corollary}{Следствие}
\newtheorem*{note}{Замечание}
\newtheorem*{reminder}{Напоминание}
\newtheorem*{example}{Пример}

\theoremstyle{remark}
\newtheorem*{solution}{Решение}

%%% Оформление страницы
\usepackage{extsizes}     % Возможность сделать 14-й шрифт
\usepackage{geometry}     % Простой способ задавать поля
\usepackage{setspace}     % Интерлиньяж
\usepackage{enumitem}     % Настройка окружений itemize и enumerate
\setlist{leftmargin=10pt} % Отступы в itemize и enumerate

\geometry{top=15mm}    % Поля сверху страницы
\geometry{bottom=5mm} % Поля снизу страницы
\geometry{left=10mm}   % Поля слева страницы
\geometry{right=10mm}  % Поля справа страницы

\setlength\parindent{15pt}        % Устанавливает длину красной строки 15pt
\linespread{1}                  % Коэффициент межстрочного интервала
%\setlength{\parskip}{0.5em}      % Вертикальный интервал между абзацами
\setcounter{secnumdepth}{0}      % Отключение нумерации разделов
%\setcounter{section}{-1}         % Нумерация секций с нуля
\usepackage{multicol}			  % Для текста в нескольких колонках
\usepackage{soulutf8}             % Модификаторы начертания
\mathtoolsset{showonlyrefs=true} % показывать номера формул только у тех, у которых есть ссылки по eqref
%%% Содержаниие
% \usepackage{tocloft}
% \tocloftpagestyle{main}
%\setlength{\cftsecnumwidth}{2.3em}
%\renewcommand{\cftsecdotsep}{1}
%\renewcommand{\cftsecpresnum}{\hfill}
%\renewcommand{\cftsecaftersnum}{\quad}

%%% Нумерация уравнений
\makeatletter
\def\eqref{\@ifstar\@eqref\@@eqref}
\def\@eqref#1{\textup{\tagform@{\ref*{#1}}}}
\def\@@eqref#1{\textup{\tagform@{\ref{#1}}}}
\makeatother                      % \eqref* без гиперссылки
\numberwithin{equation}{section}  % Нумерация вида (номер_секции).(номер_уравнения)
\mathtoolsset{showonlyrefs= true} % Номера только у формул с \eqref{} в тексте.

%%% Гиперссылки
\usepackage{hyperref}
\usepackage[usenames,dvipsnames,svgnames,table,rgb]{xcolor}
\hypersetup{
	unicode=true,            % русские буквы в раздела PDF
	colorlinks=true,       	 % Цветные ссылки вместо ссылок в рамках
	linkcolor=black!15!blue, % Внутренние ссылки
	citecolor=green,         % Ссылки на библиографию
	filecolor=magenta,       % Ссылки на файлы
	urlcolor=NavyBlue,       % Ссылки на URL
}

%%% Графика
\usepackage{tikz}        % Графический пакет tikz
\usepackage{tikz-cd}     % Коммутативные диаграммы
\usepackage{tkz-euclide} % Геометрия
\usepackage{stackengine} % Многострочные тексты в картинках
\usetikzlibrary{angles, babel, quotes}
\newcommand{\mk}{\mathfrak}
\newcommand{\s}[1]{\frac{\sigma_{#1}}{#1}}
\newcommand{\sq}[2]{\frac{\sigma_{#1}^{#2}}{#1^{#2}}}
%%% значок в itemize
\renewcommand{\labelitemi}{$\multimap  $}

\title{
  \newlength{\imageheight}
  \settoheight{\imageheight}{A}
  \texttt{Опыт Миликена\includegraphics*[height=0.7\imageheight]{10_12_22.png}\\ 3.3.3}
  }
\author{Александр Ляпин, Кирилл Нелюбин, Б05-207}
\date{\today}

\begin{document}
\maketitle
\section{Цель работы:}
Обнаружение дискретности электрического заряда
\section{В работе используются:}  
Плоский конденсатор в защитном кожухе, осветитель, измерительный микроскоп, выпрямитель, электростатический вольтметр, переключатель напряжения, пульверизатор с маслом, секундомер.

\section*{\texttt{Важные формулы}}\noindent
Формула заряда капли:
\begin{align}
  q=9\pi\sqrt{\frac{2 \eta^3 h^3}{\rho g}}
    \frac{t_0+t}{Vt_0^{3/2}t}
    \label{q}
\end{align}
Погрешность:
\begin{align}
	\s{q} =\sqrt{
    \sq V 2 + \frac 1 4\sq t 2 + \sq{t_0} 2
  }
    \label{sigma_q}
\end{align}



\section*{Экспериментальная установка}
\begin{figure}[H]
  \includegraphics*[width=0.8\textwidth]{10_10_26.png}
  \label{ust}
  \caption{Схема экспериментальной установки для измерения заряда электрона}
\end{figure}


\section{Ход работы}
\begin{enumerate}
  \setcounter{enumi}{-1}
  \item Вычислим из формулы для заряда капли (в начаде), $V\thickapprox 1$кВ.
  \item Включим осветитель, настроим окуляр на резкое изборажение.
  \item Аккуратно нажмём на пульверизатор, выберем 1 каплю.
  \item С помощью секундомера засечём время её падения на 1 мм, включим напряжение, засечём время её падения на 1 мм при включенном напряжении, запишем напряжение. Для каждой капли повторим 3-5 раз для уменьшения случайной погрешности измерения времени.
  \item Повторим предыдущий пункт для 14 капель, результаты сведём в таблицу (в приложении)
  \item Вычислим заряды капель:
  \begin{figure}[h!]
    \includegraphics*[width=0.95\textwidth]{17_14_01.png}
    \caption*{заряды капель, $\cdot 10^{-19}$Кл}
  \end{figure}
  \item Разобьём на группы, возьмём в каждой медиану, поделим каждую медиану на первую:
  \begin{figure}[h!]
    \begin{tabular}{r}
      1.00 \\
      3.80 \\
      6.76 \\
      7.41 \\
      8.19 \\
      \end{tabular}
      \caption*{\texttt{Заряды капель к наименьшему заряду}}
  \end{figure}
  \item Оценим максимальный путь релаксации по формуле:
  $$s \sim \frac{1}{g} \left( \frac{h}{t_0}\right)^2$$
  \underline{s = $8.2\cdot 10^{-2}$нм}.
\end{enumerate}


\section*{Вывод}
Предположение о дискретности заряда не подтверждается. Помешало наблюдение броуновское движение капель (можно убдедиться, исследовав разборс $t$ для отдельной капли)\\
Путь релаксации: s = $8.2\cdot 10^-2$нМ.

\newpage
\section*{\texttt{Приложение}}
\noindent
\begin{figure}[h!]
  \begin{tabular}{rrrr}
\toprule
 n &    t0 &     t &    kV \\
\midrule
 1 & 15.65 & 13.25 &   480 \\
 1 & 16.33 & 14.13 &   480 \\
 1 & 14.16 & 14.13 &   460 \\
 1 & 17.38 & 15.25 &   460 \\
 1 & 15.85 & 13.51 &   490 \\
 3 & 23.74 &  9.11 &   470 \\
 3 & 22.00 &  9.20 &   460 \\
 3 & 21.76 &  7.83 &   470 \\
 3 & 21.00 &  8.05 &   470 \\
 3 & 21.08 &  8.74 &   480 \\
 4 & 21.88 &  8.38 &   470 \\
 4 & 21.18 &  7.55 &   470 \\
 4 & 24.73 &  7.23 &   470 \\
 4 & 25.38 &  8.15 &   470 \\
 4 & 23.76 &  8.26 &   470 \\
 5 & 20.25 &  8.25 &  2000 \\
 5 & 21.00 &  7.71 &  2000 \\
 5 & 19.00 &  8.36 &  2000 \\
 5 & 18.45 &  8.48 &  2000 \\
 5 & 18.73 &  8.06 &  2000 \\
 6 & 16.13 & 15.25 & -2200 \\
 6 & 18.03 & 15.48 & -2200 \\
 6 & 16.21 & 16.81 & -2200 \\
 6 & 16.30 & 16.53 & -2200 \\
 6 & 15.38 & 15.56 & -2200 \\
 7 & 13.35 & 15.56 &  -290 \\
 7 & 12.55 & 15.06 &  -290 \\
 7 & 12.51 & 14.66 &  -290 \\
 7 & 12.60 & 16.09 &  -280 \\
 7 & 13.88 & 14.78 &  -280 \\
\bottomrule
\end{tabular}

  \begin{tabular}{rrrr}
\toprule
 n &    t0 &     t &   kV \\
\midrule
 8 & 25.68 &  9.73 & -200 \\
 8 & 23.81 &  9.23 & -200 \\
 8 & 22.65 & 10.30 & -200 \\
 8 & 23.21 &  9.78 & -200 \\
 8 & 22.18 &  5.91 & -300 \\
 9 & 16.46 & 14.83 &  220 \\
 9 & 15.08 & 13.78 &  220 \\
 9 & 17.80 & 14.83 &  220 \\
 9 & 15.73 & 15.30 &  220 \\
 9 & 16.48 & 13.90 &  220 \\
10 & 19.20 & 13.80 & -340 \\
10 & 21.13 & 13.60 & -340 \\
10 & 21.48 & 15.23 & -340 \\
10 & 20.11 & 15.46 & -340 \\
10 & 20.93 & 14.18 & -340 \\
11 & 19.13 & 18.51 &  340 \\
11 & 17.46 & 18.11 &  340 \\
11 & 17.26 & 18.35 &  340 \\
11 & 18.38 & 16.53 &  340 \\
11 & 16.56 & 21.65 &  340 \\
12 & 11.81 & 11.56 & -340 \\
12 & 12.15 & 12.83 & -340 \\
12 & 11.85 & 11.16 & -340 \\
12 & 11.80 & 11.75 & -340 \\
12 & 12.00 & 11.56 & -340 \\
13 & 35.35 & 13.28 &  250 \\
13 & 29.95 & 14.75 &  250 \\
13 & 29.91 & 12.46 &  250 \\
14 & 13.05 &  9.00 & -340 \\
14 & 14.18 & 20.53 & -220 \\
14 & 12.71 & 21.00 & -220 \\
\bottomrule
\end{tabular}

  % \label{hui}
  \caption*{Измерение капель}
\end{figure}

\end{document}